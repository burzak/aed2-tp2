\section{colaPrioridadMin($\sigma$)}

\textbf{parámetros formales}\hangindent=2\parindent\\
\parbox{1.7cm}{\textbf{géneros}}  $\sigma$\\
\parbox[t]{1.7cm}{\textbf{función}}\parbox[t]{\textwidth-2\parindent-1.7cm}{%
\InterfazFuncion{$\bullet$=$\bullet$}{\In{s1, s2}{$\sigma$}}{$\sigma$}
{$res \igobs s1 = s2$}
[$O(1)$]
[función de igualdad de $\sigma$'s]
}
$\newline$
$\newline$
$\newline$
\textbf{Se explica con: Cola de Prioridad($\alpha$), Iterador Unidireccional Modificable ($\alpha$).}
$\newline$
\textbf{Géneros: colaPrioridadMin($\sigma$), colaMin($\sigma$), iterCola($\sigma$).}


\subsection{Interfaz}

\ttitulo{Operaciones del modulo}

\InterfazFuncion{Vacia}{}{colaPrioridadMin($\sigma$)}
{res = vacia}
[O(1)]
[Crea una cola de prioridad vacia]\\

\InterfazFuncion{EsVacia?}{\In {c}{colaPrioridadMin($\sigma$)}}{bool}
{$res \igobs vacia?(c)$}
[O(1)]
[Devuelve verdadero si la lista es vacia.]\\

\InterfazFuncion{Proximo}{\In {c}{colaPrioridadMin($\sigma$)}}{$\sigma$}
[$\neg EsVacia?(x)$]
{$res \igobs proximo(c)$}
[O(1)]
[Devuelve el valor del primer elemento de la cola de prioridad]\\

\InterfazFuncion{Encolar}{\Inout {c}{colaPrioridadMin($\sigma$)}, \In {k}{$\kappa$}, \In {s}{$\beta$}}{iterColaMin}
{$res \igobs encolar(x,c)$}
[O(log(n))]
[Agrega un elemento a la cola de prioridad. Devuelve un iterador apuntando al elemento insertado]\\

\InterfazFuncion{Desencolar}{\Inout {c}{colaPrioridadMin($\sigma$)}}{}
[$\neg EsVacia?(x)$]
{$res \igobs desencolar(c)$}
[O(log(n))]
[Elimina el proximo elemento de la cola de prioridad.]\\

\InterfazFuncion{Borrar}{\Inout {c}{colaPrioridadMin($\sigma$)}, \In {it}{iterCola}}{}
[c $\igobs$ c$_0$ ]
{c $\igobs$ Borrar(c, it)}
[O(log n), siendo n la cantidad de elementos de la cola]
[Borrar el elemento apuntado por el iterador, y rearma la cola]\\

\ttitulo{Operaciones del iterador}

\InterfazFuncion{CrearIt}{\In {c}{colaMin}}{iterCola}
{res $\igobs$ crearItMod(<>, c) $\wedge$ alias(SecuSuby(it) = l)}
[O(1)]
[Crea un iterador a la cola a la raiz de la cola]\\

\InterfazFuncion{HayMas?}{\In {it}{iterCola}}{bool}
{res $\igobs$ HayMas?(it)}
[O(1)]
[Chequea que el iterador este apuntando a un elemento]\\

\InterfazFuncion{Actual}{\In {it}{iterCola}}{nat}
[HayMas?(it)]
{res $\igobs$ Actual(it)}
[O(1)]
[Devuelve el jugador apuntado por el iterador]\\

\begin{Representacion}

\subsection{Justificacion}

\ttitulo{Representacion del modulo}

Implementamos la cola de prioridad sobre minHeap para poder tener acceso en O(1) al primer de elemento de la cola y tener las operaciones de encolar y desencolar en O(log(n)). $\newline$ Como queremos poder agregar y quitar una arbitraria cantidad de elementos, no podemos usar un arreglo para representar el heap, dado que para extender el arreglo tendríamos una complejidad mayor a la pedida. $\newline$ Lo representamos sobre un árbol binario completo izquierdista, cumpliendo la complejidad de heap. $\newline$ El principal uso de la cola de prioridad será para determinar qué jugador captura al pokemon, la prioridad será: aquel que tenga la menor cantidad de pokemons tendrá la mayor prioridad. $\newline$ No nos interesa buscar en la estructura de manera eficiente (la complejidad de la búsqueda es O(n)) pero sí nos interesa obtener el primer elemento en O(1). $\newline$
	\begin{Estructura}{colaPrioridadMin($\sigma$)}[colaMin]
		
		\begin{Tupla}[colaMin]
			\tupItem{raiz}{puntero(nodoHeap($\sigma$))}
			\tupItem{ultimoAgregdo}{puntero(nodoHeap($\sigma$)))}		
		\end{Tupla}
		
		\begin{Tupla}[nodoHeap]
			\tupItem{padre}{puntero(nodoHeap($\sigma$))}
			\tupItem{izq}{puntero(nodoHeap($\sigma$))}
			\tupItem{der}{puntero(nodoHeap($\sigma$))}
			\tupItem{clave}{$\kappa$}
			\tupItem{significado}{$\beta$}
		\end{Tupla}
	\end{Estructura}
\subsection{Invariante de representación}

\textbf{Informal}\\
(1)Para todo nodo, sus hijos no pueden tenerlo como hijo.\\
(2)La raíz no tiene padre.\\
(3)Árbol binario perfectamente balanceado e izquierdista.\\
(4)La clave de cada nodo es menor o igual a la de sus hijos (si los tiene).\\
(5)Todo subárbol es un heap.\\

\textbf{Formal}
\Rep[colaMin C][C]{C.raiz = NULLL $\vee$ ((2) C.raiz->padre = NULL $\wedge$ (3) ($\forall$ p: nodoHeap($\sigma$)) p.der $\neq$ NULL $\Rightarrow$ p.izq $\neq$ NULL $\wedge$ (4) ($\forall$ p: nodoHeap($\sigma$))(((p.izq $\neq$ NuLL) $\impluego$ p.clave $>$ p.izq->clave) $\wedge$ ((p.der $\neq$ NULL) $\impluego$ p.clave $>$ p.der->clave)) $\wedge$ (5) Rep(C.raiz->izq) $\wedge$ Rep(C.raiz->der))}


\subsection{Predicado de abstracción}

\AbsFc[colaM]{colaMin}[C]{ cola :Cola($\sigma$)| EsVacia?(C) = vacia(cola) $\oluego$ (Proximo(C) = proximo(cola) $\wedge$ Desencolar(C) = desencolar(cola))}

\subsection{Representación del iterador}

\subsection{Justificación}

El iterador nos permite acceder a cualquier elemento del heap en tiempo constante, sin necesidad de hacer búsqueda exhaustiva. En particular nos interesa eliminar un elemento (cuando un jugador se aleja del pokemon) en tiempo O(log(n)).
Este iterador en particular no tiene función avanzar porque sólo nos interesa que apunte siempre al mismo nodo.


	\begin{Estructura}{iterCola}[iter]
		\begin{Tupla}[iter]
			\tupItem{Siguiente}{puntero(nodoHeap)}
		\end{Tupla}
	\end{Estructura}

\subsection{Invariante de representación}
\ttitulo{Informal} $\newline$
(1) True \\

\ttitulo{Formal}

\Rep[iterCola It][It]{true}

\subsection{Predicado de abstraccion}

\AbsFc[iter]{iterCola}[I]{ it :iterador Unidireccional| Actual(I) $\bullet$ <> = Siguientes(it)}

\subsection{Algoritmos}

\ttitulo{Funciones Auxiliares, no exportadas}

\InterfazFuncion{BuscarPadreInsertar}{\Inout {c}{colaPrioridadMin}}{}
{Conserva la propiedad de heap}
[O(log(n))]
[Encuentra el nodo al cual hay que insertarle el próximo hijo]\\

\InterfazFuncion{SiftUp}{\Inout {c}{colaPrioridadMin}, \In {p}{puntero}}{}
[$\neg EVacia?(Desencolar(c))$]
{No necesariamente conserva la propiedad de heap}
[O(1)]
[Mueve hacia arriba en la cola de prioridad a p si su padre es de menor prioridad]\\

\InterfazFuncion{SiftDown}{\Inout {c}{colaPrioridadMin}, \In {p}{puntero}}{}
[$\neg EVacia?(Desencolar(c))$]
{No necesariamente conserva la propiedad de heap}
[O(1)]
[Mueve hacia abajo en la cola de prioridad a p si alguno de sus hijos es de mayor prioridad]\\

\InterfazFuncion{Swap}{\Inout {c}{colaPrioridadMin}, \In {p}{puntero}, \In {q}{puntero}}{}
[$\neg EVacia?(Desencolar(c))$]
{No necesariamente conserva la propiedad de heap}
[O(1)]
[Toma dos nodos del heap e intercambia sus posiciones.]\\

\InterfazFuncion{TieneQueSubir?}{\Inout {c}{colaPrioridadMin}, \In {p}{puntero}}{bool}
[$\neg EVacia?(Desencolar(c))$]
{No necesariamente conserva la propiedad de heap}
[O(1)]
[Verifica si hay que hacer siftUp para reestablecer la propiedad del heap]\\

\InterfazFuncion{TieneQueBajar?}{\Inout {c}{colaPrioridadMin}, \In {p}{puntero}}{bool}
[$\neg EVacia?(Desencolar(c))$]
{No necesariamente conserva la propiedad de heap}
[O(1)]
[Verifica si hay que hacer siftDown para reestablecer la propiedad del heap]\\

\InterfazFuncion{Borrar}{\Inout {c}{colaPrioridadMin}, \In {p}{puntero}}{}
[$\neg EVacia?(c)$]
{No necesariamente conserva la propiedad de heap}
[O(1)]
[Elimina un nodo del árbol]\\

\InterfazFuncion{Eliminar}{\Inout {it}{iterCola}}{}
[HayMas?(it) $\wedge$ it $=$ it$_0$]
{it $=$ Eliminar(it$_0$)}
[O(1)]
[Borra el nodo apuntado por el iterador y pone el iterador a NULL]\\



\begin{Algoritmos}

\begin{algorithm}[H]{\textbf{iVacia}() $\to$ res: colaMin}
	\begin{algorithmic}[1]
		%\State a $\gets$ puntero(nodoHeap($\sigma$)) \Comment O(1), creo un nodo vacío
		\State $res \gets <NULL, NULL>$ \Comment O(1)
		
		\medskip
		\Statex \underline{Complejidad:} O(1)
			\Statex \underline{Justificacion:} Crea un heap vacio, como siempre va a tener que crear una raíz vacía, la complejidad queda constante.
	\end{algorithmic}
\end{algorithm}


\begin{algorithm}[H]{\textbf{iEsVacia?}(\In {r}{colaMin}) $\to$ res: bool)}
	\begin{algorithmic}[1]
		\State $res \gets r.raiz = NULL$\Comment O(1)
		
		\medskip
		\Statex \underline{Complejidad:} O(1)
			\Statex \underline{Justificacion:} Hace una comparación con la raíz para verificar que la cola de prioridad esté vacía.
	\end{algorithmic}
\end{algorithm}

\begin{algorithm}[H]{\textbf{iEncolar}(\Inout {c}{colaMin}, \In {k}{$\kappa$}, \In {s}{$\beta$}) $\to$ res: iterColaMin}
	\begin{algorithmic}[1]

		\State puntero(nodoHeap($\sigma$)) insertado
		
		\If{c.raiz = NULL}
			\State c.raiz$\gets$ insertado
			\State c.ultimo $\gets$ insertado
		\EndIf
		
		\State dondeAgregar $\gets$ buscarPadreInsertar(c, ultimo)
		
		\If{dondeAgregar$\to$izq = NULL}
			\State dondeAgrega$\to$izq $\gets$ insertado
		\Else
			\If {dondeAgregar$\to$der = NULL}
				\State dondeAgregar$\to$der $\gets$ insertado
			\EndIf
		\EndIf
		
		\State insertado$\to$padre $\gets$ dondeAgregar
		\State c.ultimo $\gets$ insertado
		
		\State SiftUp(c, insertado)
		\State res $\gets$ iterador(insertado)
		
		
		\medskip
		\Statex \underline{Complejidad:} O(log(n))
			\Statex \underline{Justificacion:} Se agrega un elemento al fondo del heap, luego se utiliza la función sift-up para volver a la propiedad de heap. Ocurren varias operaciones independientes que tienen complejidad O(log(n)), por lo tanto tiene una complejidad O(log(n)).
	\end{algorithmic}
\end{algorithm}


\begin{algorithm}[H]{\textbf{iDesencolar}(\Inout {c}{colaMin}})
	\begin{algorithmic}[1]

		\If {c.raiz$\to$izq = NULL}
			\State c.raiz = NULL
			\State c.ultimo = NULL
		\Else	
			\State puntero(nodoHeap($\sigma$)) nuevoUltimo $\gets$ buscarUltimo(c.ultimo)

			\State puntero(nodoHeap($\sigma$)) aBorrar $\gets$ c.raiz
			\State swap(c.raiz, c.ultimo)
		
			\If {c.ultimo$\to$padre$\to$der = aBorrar}
				\State c.ultimo$\to$padre$\to$der $\gets$ NULL
			\Else
				\State c.ultimo$\to$padre$\to$izq $\gets$ NULL
			\EndIf
		
		\State c.ultimo $\gets$ nuevoUltimo
		\State SiftDown(c.raiz)

		\EndIf	

		
		\medskip
		\Statex \underline{Complejidad:} O(log(n))
			\Statex \underline{Justificacion:} Se elimina el primer elemento del heap, se reemplaza la raiz por el último elemento, luego se utiliza la función sift-down para volver a tener la propiedad de heap. El ciclo itera como máximo log(n) veces, que es la altura del heap (sólo se puede llevar hasta el fondo).  
	\end{algorithmic}
\end{algorithm}


\begin{algorithm}[H]{\textbf{iProximo}(\In {c}{colaMin}) $\to$ res: $\sigma$}
	\begin{algorithmic}[1]
		
		\State res$\leftarrow$c$\rightarrow$raiz.significado
		
		\medskip
		\Statex \underline{Complejidad:} O(1)
			\Statex \underline{Justificación:} Se obtiene el significado de la raiz del heap.
	\end{algorithmic}
\end{algorithm}


\begin{algorithm}[H]{\textbf{iBorrar}(\Inout {c}{colaPrioridadMin($\sigma$)}, \Inout {p}{iterCola}}
	\begin{algorithmic}[1]
		\State puntero(nodoHeap($\sigma$)) ultimo $\leftarrow$ NULL\Comment O(1)
		\,
		\If{c.padreAgregar$\rightarrow$der = NULL}\Comment O(1)
			\State ultimo$\leftarrow$ c.padreAgregar$\rightarrow$izq\Comment O(1)
			\State c.padreAgregar$\rightarrow$izq$\leftarrow$NULL\Comment O(1)
			\,
		\Else
			\State ultimo $\leftarrow$ c.padreAgregar$\rightarrow$der\Comment O(1)
			\State c.padreAgregar$\rightarrow$der$\leftarrow$NULL\Comment O(1)
		\EndIf
		\State ultimo$\rightarrow$izq$\leftarrow$p$\rightarrow$izq\Comment O(1)
		\State ultimo$\rightarrow$der$\leftarrow$p$\rightarrow$der\Comment O(1)
		\State ultimo$\to$padre$\leftarrow$p$\to$padre	\Comment O(1)
		\,
		%Ahora sift down
		\While {ultimo$\rightarrow$der $\neq$ NULL $\wedge$ ultimo$\rightarrow$izq $\neq$ NULL $\wedge$ (ultimo$\rightarrow$der.clave $<$ ultimo.clave $\vee$ ultimo$\rightarrow$izq.clave $<$ ultimo.clave))} \Comment O(log(n))
			\State siftDown(ultimo, c)\Comment O(1)
		\EndWhile
		\State Eleminar(p) \Comment O(1)
		\medskip
		\Statex \underline{Complejidad:} O(log(n))
			\Statex \underline{Justificacion:} Muy similar a iDesencolar(), pero eliminando el puntero del parámetro. $\newline$ Se elimina el elemento apuntado por el iterador, se reemplaza por el último elemento, luego se utiliza la función sift-down para volver a tener la propiedad de heap. El ciclo itera como máximo log(n) veces, que es la altura del heap (sólo se puede llevar hasta el fondo).  
	\end{algorithmic}
\end{algorithm}



%%%%%%%%%%%%%%%%%%%%%%%%%%%%%%
% Funciones auxiliares
%%%%%%%%%%%%%%%%%%%%%%%%%%%%%%

\begin{algorithm}[H]{\textbf{tieneQueSubir}(\In {c}{colaPrioridadMin($\sigma$)}, \In {n}{puntero(nodoHeap($\sigma$))}) $\to$ res: Bool}
	\begin{algorithmic}[1]
	
		\If {n$\to$padre = NULL}
			\State res $\gets$ False
		\Else
			\State res $\gets$ n$\to$valor < n$\to$padre$\to$valor
		\EndIf

		\medskip
		\Statex \underline{Complejidad:} O(1)
			\Statex \underline{Justificacion:}Una pequeña función para determinar si hay que hacer siftUp. 
	\end{algorithmic}
\end{algorithm}

\begin{algorithm}[H]{\textbf{tieneQueBajar}(\In {c}{colaPrioridadMin($\sigma$)}, \In {n}{puntero(nodoHeap($\sigma$))}) $\to$ res: Bool}
	\begin{algorithmic}[1]
	
		\If {n$\to$izq = NULL $\wedge$ n$\to$der = NULL}
			\State res $\gets$ False
		\Else
			\If {n$\to$der != NULL $\wedge$ n$\to$der$\to$valor < n$\to$valor}
				\State res $\gets$ True
			\Else
				\State res $\gets$ n$\to$izq$\to$valor < n$\to$valor	
			\EndIf
		\EndIf

		\medskip
		\Statex \underline{Complejidad:} O(1)
			\Statex \underline{Justificacion:}Una pequeña función para determinar si hay que hacer siftDown. 
	\end{algorithmic}
\end{algorithm}

\begin{algorithm}[H]{\textbf{swap}(\In {c}{colaPrioridadMin($\sigma$)}, \In {nodo1}{puntero(nodoHeap($\sigma$))}, \In {nodo2}{puntero(nodoHeap($\sigma$))})}
	\begin{algorithmic}[1]
		
		\State n1 $\gets$ nodo1
		\State n1Padre $\gets$ nodo1$\to$padre
		\State n1Der $\gets$ nodo1$\to$der
		\State n1Izq $\gets$ nodo1$\to$izq
		
		\State n2 $\gets$ nodo2
		\State n2Padre $\gets$ nodo2$\to$padre
		\State n2Der $\gets$ nodo2$\to$der
		\State n2Izq $\gets$ nodo2$\to$izq
		
		\Comment Cambiamos los padres
		\If {nodo1$\to$padre = nodo2$\to$padre}
			\State No hacer nada, dejarlos con el mismo padre
		\Else
			\If {nodo1$\to$padre != nodo2 $\wedge$ nodo2$\to$padre != nodo1}
				\State nodo1$\to$padre $\gets$ n2Padre	
				\State nodo2$\to$padre $\gets$ n1Padre
				\If {nodo1$\to$izq != NULL}
					\State nodo1$\to$der$\to$padre $\gets$ nodo2
				\EndIf
				\If {nodo1$\to$der != NULL}
					\State nodo1$\to$izq$\to$padre $\gets$ nodo2
				\EndIf
				\If {nodo2$\to$izq != NULL}
					\State nodo1$\to$izq$\to$padre $\gets$ nodo1
				\EndIf
				\If {nodo2$\to$der != NULL}
					\State nodo1$\to$der$\to$padre $\gets$ nodo1
				\EndIf
			\Else
				\If {nodo2$\to$padre = nodo1}
					\State nodo1$\to$padre $\gets$ n2
					\State nodo2$\to$padre $\gets$ n1Padre	
					\If {nodo1$\to$izq != NULL $\wedge$ nodo1$\to$izq != nodo2)}
						\State nodo1$\to$izq$\to$padre $\gets$ nodo2					
					\EndIf
					\If {nodo1$\to$der != NULL $\wedge$ nodo1$\to$der != nodo2}
						\State nodo1$\to$der$\to$padre $\gets$ nodo2					
					\EndIf
					\If {nodo2$\to$izq != NULL}
						\State nodo2$\to$izq$\to$padre $\gets$ nodo1
					\EndIf
					\If {nodo2$\to$der != NULL}
						\State nodo2$\to$der$\to$padre $\gets$ nodo1
					\EndIf
				\Else
					\If {nodo1$\to$padre = nodo2}
						\State nodo1$\to$padre $\gets$ n2Padre
						\State nodo2$\to$padre = n1
						\If {nodo1$\to$izq != NULL}
							\State nodo1$\to$izq$\to$padre $\gets$ nodo2
						\EndIf
						\If {nodo1$\to$der != NULL}
							\State nodo1$\to$der$\to$padre $\gets$ nodo2
						\EndIf
						\If {nodo2$\to$izq != NULL}
							\State nodo2$\to$izq$\to$padre $\gets$ nodo1
						\EndIf
						\If {nodo2$\to$der != NULL}
							\State nodo2$\to$der$\to$padre $\gets$ nodo1
						\EndIf
					\EndIf
				\EndIf
			
				\Comment Apuntamos los nuevos hijos de cada nodo			
				\State nodo1$\to$izq $\gets$ n2Izq
				\If {nodo1$\to$izq != NULL $\wedge$ nodo1$\to$izq = nodo1}
					\State nodo1$\to$izq $\gets$ nodo2
				\EndIf
				\State nodo1$\to$der $\gets$ n2Der
				\If {nodo1$\to$der != NULL $\wedge$ nodo1$\to$der = nodo1}
					\State nodo1$\to$der $\gets$ nodo2
				\EndIf
				\State nodo2$\to$izq $\gets$ n1Izq
				\If {nodo2$\to$izq != NULL $\wedge$ nodo2$\to$izq = nodo2}
					\State nodo2$\to$izq $\gets$ nodo1
				\EndIf
				\State nodo2$\to$der $\gets$ n1Der
				\If {nodo2$\to$der != NULL $\wedge$ nodo2$\to$der = nodo2}
					\State nodo2$\to$der $\gets$ nodo1
				\EndIf
				
				\Comment Tenemos que actualizar las referencias de los padres
				\If {nodo1$\to$padre = nodo2$\to$padre}
					\State izqAux $\gets$ nodo1$\to$padre$\to$izq
					\State nodo1$\to$padre$\to$izq $\gets$ nodo1$\to$padre$\to$der
					\State nodo1$\to$padre$\to$der $\gets$ izqAux
				\Else
					\If {n1Padre != NULL $\wedge$ n1Padre$\to$izq = n1}
						\State n1Padre$\to$izq $\gets$ n2
					\EndIf
					\If {n1Padre != NULL $\wedge$ n1Padre$\to$der = n1}
						\State n1Padre$\to$der $\gets$ n2
					\EndIf
					\If {n2Padre != NULL $\wedge$ n2Padre$\to$izq = n2}
						\State n2Padre$\to$izq $\gets$ n1
					\EndIf
					\If {n2Padre != NULL $\wedge$ n2Padre$\to$der = n2}
						\State n2Padre$\to$der $\gets$ n1
					\EndIf				
				\EndIf
			\EndIf
			
			\Comment Seteamos la nueva raíz
			\If {nodo1 = c$\to$raiz}			
				\State c$\to$raiz $\gets$ nodo2			
			\Else
				\If {nodo2 = c$\to$raiz}
					\State c$\to$raiz $\gets$ nodo1
				\EndIf			
			\EndIf
			
			\Comment Actualizamos el ultimo
			\If {c$\to$ultimo = nodo1}
				\State c$\to$ultimo $\gets$ nodo2
			\Else
				\If {c$\to$ultimo = nodo2}
					\State c$\to$ultimo $\gets$ nodo1			
				\EndIf
			\EndIf
		\EndIf
		\medskip
		\Statex \underline{Complejidad:} O(1)
			\Statex \underline{Justificacion:}Una pequeña función para determinar si hay que hacer siftDown. 
	\end{algorithmic}
\end{algorithm}


\begin{algorithm}[H]{\textbf{BuscarPadreInsertar}(\In {c}{colaPrioridadMin($\sigma$)})}
	\begin{algorithmic}[1]
	
		\If {c.padreAgregar$\rightarrow$der $\neq$ NULL}\Comment O(1)
			\State puntero(nodoHeap($\sigma$)) aux $\leftarrow$ c.padreAgregar\Comment O(1)
			\,	
			\While {aux $\neq$ c.raiz $\wedge$ aux$\rightarrow$padre$\rightarrow$der = aux}\Comment O(log(n))\,
				\Comment Buscamos el padre hacia la izquierda hasta no poder avanzar
				\State aux $\leftarrow$ aux$\rightarrow$padre\Comment O(1)		
			\EndWhile
			\,			
			\If {aux $\neq$ c.raiz}\Comment O(1)\,
				\Comment Nos movemos al subárbol hermano
				\State aux $\leftarrow$ aux$\rightarrow$padre\Comment O(1)
				\State aux $\leftarrow$ aux$\rightarrow$der\Comment O(1)
			\EndIf
			\,			
			\While {aux$\rightarrow$izq $\neq$ NULL}\Comment O(1)\,
				\Comment Bajamos a la izquierda hasta no poder avanzar.
				\State aux $\leftarrow$ aux$\rightarrow$izq\Comment O(1)
			\EndWhile
			\,
			\State c.padreAgregar $\leftarrow$ aux\Comment O(1)
		\EndIf
		
		\medskip
		\Statex \underline{Complejidad:} O(log(n))
			\Statex \underline{Justificacion:}Se obtiene nodo al cual hay que agregarle hojas para mantener la propiedad de heap al encolar.\,
			     Subimos hacia la izquierda mientras los subárboles estén completos para encontrar el subárbol "hermano" al cual hay que agregarle el elemento a encolar. Luego bajamos hacia la izquierda para encontrar el padre de la futura hoja. En caso de que el árbol total sea completo se sube hasta la raíz y se baja hacia el elemento más a la izquierda.
			\,   En el peor caso se recorre dos veces la altura del heap, es decir 2log(n) operaciones, dando una complejidad O(log(n)). 
	\end{algorithmic}
\end{algorithm}

\begin{algorithm}[H]{\textbf{siftUp}(\In {c}{colaPrioridadMin($\sigma$)}, \In {n}{puntero(nodoHeap($\sigma$))})}
	\begin{algorithmic}[1]

		\While {tieneQueSubir(n)}
			\State swap(n, n$\to$padre)		
		\EndWhile
		\medskip
		\Statex \underline{Complejidad:} O(1)
			\Statex \underline{Justificacion:}Se intercambia un nodo con uno de sus hijos que tenga clave menor. Pese a que no tiene precondiciones, es llamado en un contexto en el cual no se puede indefinir. Es una secuencia de asignaciones, todas O(1), por lo tanto el algoritmo tiene complejidad O(1).
	\end{algorithmic}
\end{algorithm}

\begin{algorithm}[H]{\textbf{siftDown}(\In {c}{colaPrioridadMin($\sigma$)}, \In {n}{puntero(nodoHeap($\sigma$))})}
	\begin{algorithmic}[1]
		
		\State aBajar $\gets$ nodoEvaluado
		\While {tieneQueBajar(aBajar)}
			\If {aBajar$\to$der = NULL}
				\State swap(aBajar$\to$izq, aBajar)
			\Else
				\If {aBajar$\to$izq != NULL $\wedge$ aBajar$\to$der != NULL}
					\If {aBajar$\to$izq$\to$valor < aBajar$\to$der$\to$valor}
						\State swap(aBajar$\to$izq, aBajar)					
					\Else
						\State swap(aBajar$\to$der, aBajar)
					\EndIf	
				\EndIf
			\EndIf
		\EndWhile		

		\medskip
		\Statex \underline{Complejidad:} O(1)
			\Statex \underline{Justificacion:}Se intercambia un nodo con su padre. Pese a que no tiene precondiciones, es llamado en un contexto en el cual no se puede indefinir. Es una secuencia de asignaciones, todas O(1), por lo tanto el algoritmo tiene complejidad O(1). 
	\end{algorithmic}
\end{algorithm}

\ttitulo{Algoritmos del iterador}

\begin{algorithm}[H]{\textbf{iCrearIt}(\In {c}{colaMin}) $\to$ res: iter}
	\begin{algorithmic}[1]
		\State res$\gets$ <c.raiz> \Comment O(1)
		\medskip
		\Statex \underline{Complejidad:} O(1)
			\Statex \underline{Justificación:}Creo un iterador que apunta a raiz de la cola
	\end{algorithmic}
\end{algorithm}

\begin{algorithm}[H]{\textbf{iEliminar}(\Inout {it}{iter})}
	\begin{algorithmic}[1]
		\State it.siguiente$\to$ padre $\gets$ NULL \Comment O(1)		
		\State it.siguiente$\to$ izq $\gets$ NULL \Comment O(1)		
		\State it.siguiente$\to$ der $\gets$ NULL \Comment O(1)		
		\State it.siguiente $\gets$ NULL \Comment O(1)		
		\medskip
		\Statex \underline{Complejidad:} O(1)
			\Statex \underline{Justificación:}Primero me con el iterador borro la relación con el padre y de los hijos, despues pongo a NULL al iterador para eliminar un nodo, nota esta funcion nunca sucede por si sola, siempre es llamada desde Borrar de la cola
	\end{algorithmic}
\end{algorithm}

\begin{algorithm}[H]{\textbf{iHayMas?}(\In {it}{iter}) $\to$ res: bool}
	\begin{algorithmic}[1]
		\State res $\gets$ it.siguiente $\neq$ NULL \Comment O(1)
		\medskip
		\Statex \underline{Complejidad:} O(1)
			\Statex \underline{Justificación:} Veo si el iterador esta apuntando a algo valido
	\end{algorithmic}
\end{algorithm}

\begin{algorithm}[H]{\textbf{iActual}(\In {it}{iter}) $\to$ res: nat}
	\begin{algorithmic}[1]
		\State res $\gets$ it.siguiente->significado \Comment O(1)		
		\medskip
		\Statex \underline{Complejidad:} O(1)
			\Statex \underline{Justificación:}Devuelve la ID del jugador al que apunta el iterador
	\end{algorithmic}
\end{algorithm}


\end{Algoritmos}

\end{Representacion}
