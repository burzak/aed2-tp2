% % Apunte de modulos basicos
%
\documentclass[a4paper,10pt]{article}
\usepackage[paper=a4paper, hmargin=1.5cm, bottom=1.5cm, top=3.5cm]{geometry}
\usepackage[T1]{fontenc}
\usepackage{fancyhdr}
\usepackage{lastpage}
\usepackage{xspace}
\usepackage{xargs}
\usepackage{ifthen}
\usepackage{aed2-symb,aed2-itef,aed2-tad,caratula}
%\usepackage{aed2-tad,aed2-symb,aed2-itef}
\usepackage{algorithmicx, algpseudocode, algorithm}
\usepackage[colorlinks=true, linkcolor=blue]{hyperref}
\usepackage[utf8]{inputenc}


\hypersetup{%
 % Para que el PDF se abra a página completa.
 pdfstartview= {FitH \hypercalcbp{\paperheight-\topmargin-1in-\headheight}},
 pdfauthor={Cátedra de Algoritmos y Estructuras de Datos II - DC - UBA},
 pdfkeywords={Módulos básicos},
 pdftitle={Módulos básicos de diseño},
 pdfsubject={Módulos básicos de diseño}
}

%%%%%%%%%%%%%%%%%%%%%%%%%%%%%%%%%%%%%%%%%%%%%%%%%
% PARAMETROS A SER MODIFICADOS
%%%%%%%%%%%%%%%%%%%%%%%%%%%%%%%%%%%%%%%%%%%%%%%%%

%cuatrimestre de acuerdo a la opcion
\newcommand{\Cuatrimestre}{$1^\mathrm{er}$ cuatrimestre de 2016}
%\newcommand{\Cuatrimestre}{$1^\mathrm{er}$ cuatrimestre de 2012 (compilado 08/05/2012)}
%\renewcommand{\Cuatrimestre}{$2^\mathrm{do}$ cuatrimestre de 2010 (compilado 13/05/2011)}

%%%%%%%%%%%%%%%%%%%%%%%%%%%%%%%%%%%%%%%%%%%%%%%%%
% OTRAS OPCIONES QUE NO HAY QUE MODIFICAR
%%%%%%%%%%%%%%%%%%%%%%%%%%%%%%%%%%%%%%%%%%%%%%%%%

% Acomodo fancyhdr.
\pagestyle{fancy}
\thispagestyle{fancy}
\lhead{Algoritmos y Estructuras de Datos II}
\rhead{\Cuatrimestre}
\cfoot{\thepage /\pageref{LastPage}}
\renewcommand{\footrulewidth}{0.4pt}
\setlength{\headheight}{13pt}

%%%%%%%%%%%%%%%%%%%%%%%%%%%%%%%%%%%%%%%%%%%%%%%%%%%%%%%%%%%%
% COMANDOS QUE ALGUN DIA PUEDAN FORMAR UN PAQUETE.
%%%%%%%%%%%%%%%%%%%%%%%%%%%%%%%%%%%%%%%%%%%%%%%%%%%%%%%%%%%%
\newcommand{\moduloNombre}[1]{\textbf{#1}}

\let\NombreFuncion=\textsc
\let\TipoVariable=\texttt
\let\ModificadorArgumento=\textbf
\newcommand{\res}{$res$\xspace}
\newcommand{\tab}{\hspace*{7mm}}

\newcommandx{\TipoFuncion}[3]{%
  \NombreFuncion{#1}(#2) \ifx#3\empty\else $\to$ \res\,: \TipoVariable{#3}\fi%
}
\newcommand{\In}[2]{\ModificadorArgumento{in} \ensuremath{#1}\,: \TipoVariable{#2}\xspace}
\newcommand{\Out}[2]{\ModificadorArgumento{out} \ensuremath{#1}\,: \TipoVariable{#2}\xspace}
\newcommand{\Inout}[2]{\ModificadorArgumento{in/out} \ensuremath{#1}\,: \TipoVariable{#2}\xspace}
\newcommand{\Aplicar}[2]{\NombreFuncion{#1}(#2)}

\newlength{\IntFuncionLengthA}
\newlength{\IntFuncionLengthB}
\newlength{\IntFuncionLengthC}
%InterfazFuncion(nombre, argumentos, valor retorno, precondicion, postcondicion, complejidad, descripcion, aliasing)
\newcommandx{\InterfazFuncion}[9][4=true,6,7,8,9]{%
  \hangindent=\parindent
  \TipoFuncion{#1}{#2}{#3}\\%
  \textbf{Pre} $\equiv$ \{#4\}\\%
  \textbf{Post} $\equiv$ \{#5\}%
  \ifx#6\empty\else\\\textbf{Complejidad:} #6\fi%
  \ifx#7\empty\else\\\textbf{Descripción:} #7\fi%
  \ifx#8\empty\else\\\textbf{Aliasing:} #8\fi%
  \ifx#9\empty\else\\\textbf{Requiere:} #9\fi%
}

\newenvironment{Interfaz}{%
  \parskip=2ex%
  \noindent\textbf{\Large Interfaz}%
  \par%
}{}

\newenvironment{Representacion}{%
  \vspace*{2ex}%
  \noindent\textbf{\Large Representación}%
  \vspace*{2ex}%
}{}

\newenvironment{Algoritmos}{%
  \vspace*{2ex}%
  \noindent\textbf{\Large Algoritmos}%
  \vspace*{2ex}%
}{}


\newcommand{\Titulo}[1]{
  \vspace*{1ex}\par\noindent\textbf{\large #1}\par
}

\newenvironmentx{Estructura}[2][2={estr}]{%
  \par\vspace*{2ex}%
  \TipoVariable{#1} \textbf{se representa con} \TipoVariable{#2}%
  \par\vspace*{1ex}%
}{%
  \par\vspace*{2ex}%
}%

\newboolean{EstructuraHayItems}
\newlength{\lenTupla}
\newenvironmentx{Tupla}[1][1={estr}]{%
    \settowidth{\lenTupla}{\hspace*{3mm}donde \TipoVariable{#1} es \TipoVariable{tupla}$($}%
    \addtolength{\lenTupla}{\parindent}%
    \hspace*{3mm}donde \TipoVariable{#1} es \TipoVariable{tupla}$($%
    \begin{minipage}[t]{\linewidth-\lenTupla}%
    \setboolean{EstructuraHayItems}{false}%
}{%
    $)$%
    \end{minipage}
}

\newcommandx{\tupItem}[3][1={\ }]{%
    %\hspace*{3mm}%
    \ifthenelse{\boolean{EstructuraHayItems}}{%
        ,#1%
    }{}%
    \emph{#2}: \TipoVariable{#3}%
    \setboolean{EstructuraHayItems}{true}%
}

\newcommandx{\RepFc}[3][1={estr},2={e}]{%
  \tadOperacion{Rep}{#1}{bool}{}%
  \tadAxioma{Rep($#2$)}{#3}%
}%

\newcommandx{\Rep}[3][1={estr},2={e}]{%
  \tadOperacion{Rep}{#1}{bool}{}%
  \tadAxioma{Rep($#2$)}{true \ssi #3}%
}%

\newcommandx{\Abs}[5][1={estr},3={e}]{%
  \tadOperacion{Abs}{#1/#3}{#2}{Rep($#3$)}%
  \settominwidth{\hangindent}{Abs($#3$) \igobs #4: #2 $\mid$ }%
  \addtolength{\hangindent}{\parindent}%
  Abs($#3$) \igobs #4: #2 $\mid$ #5%
}%

\newcommandx{\AbsFc}[4][1={estr},3={e}]{%
  \tadOperacion{Abs}{#1/#3}{#2}{Rep($#3$)}%
  \tadAxioma{Abs($#3$)}{#4}%
}%


\newcommand{\DRef}{\ensuremath{\rightarrow}}

\begin{document}

%pagina de titulo
\thispagestyle{empty}
\newpage

% Estos comandos deben ir antes del \maketitle
\materia{Algoritmos y Estructuras de Datos II} % obligatorio
\submateria{Segundo Cuatrimestre de 2016} % opcional
\titulo{Trabajo Práctico 2} % obligatorio
\subtitulo{Diseño} % opcional
\grupo{Grupo } % opcional 

\integrante{Ocles Garcia, Nestor Dario}{633/15}{dario.ocles@gmail.com} % obligatorio 
\integrante{Ansaldi, Nicolas}{128/14}{nansaldi611@gmail.com} % obligatorio 
\integrante{Pawlow, Dante}{449/12}{dante.pawlow@gmail.com} % obligatorio 

\maketitle

% compilar 2 veces para actualizar las referencias
%\tableofcontents

\pagebreak


\tableofcontents
\newpage
\section{Coordenada}


\subsection{Interfaz}

\parbox {1,7cm}{{\bf Género}} Coordenada \\
{\bf se explica con:}  \tadNombre{nat}, \tadNombre{bool}\\
\medskip

\InterfazFuncion{CrearCoordenada}{\In {n}{nat}, \In{m}{nat}}{coordenada}
{res = crerCoor(n, m)}
[O(1)]
[Crea una nueva coordenada]
\\

\InterfazFuncion{DistEuclidea}{\In {c_1}{coordenada}, \In {c_2}{coordenada}}{nat}
{res = distEuclidea($c_1$, $c_2s$)}
[O(1)]
[Devuelve la distancia entre 2 coordenadas]
\\

\InterfazFuncion{CoordenadaArriba}{\In {c}{coordenada}}{coordenada}
{res = coordenadaArriba(c)}
[O(1)]
[Crea una coordenada arriba de la pasada por parámetro]
\\

\InterfazFuncion{CoordenadaAbajo}{\In{c}{coordenada}}{coordenada}
[latitud(c)$>$0]
{res = coordenadaAbajo(c)}
[O(1)]
[Crea una coordenada abajo de la pasada por parámetro]
\\

\InterfazFuncion{CoordenadaIzquierda}{\In{c}{coordenada}}{coordenada}
[longitud(c)$>$0]
{res = coordenadaIzquierda(c)}
[O(1)]
[Crea una coordenada a la izquierda de la pasada por parámetro]
\\

\InterfazFuncion{CoordenadaDerecha}{\In{c}{coordenada}}{coordenada}
{res = coordenadaDerecha(c)}
[O(1)]
[Crea una coordenada a la derecha de la pasada por parámetro]
\\

\InterfazFuncion{TieneCoordenadaAbajo}{\In{c}{coordenada}}{bool}
{res = latitud(c) $>$ 0}
[O(1)]
[Dice si tiene coordenada abajo]
\\

\InterfazFuncion{TieneCoordenadaIzquierda}{\In{c}{coordenada}}{bool}
{res = longitud(c) $>$ 0}
[O(1)]
[Dice si tiene coordenada a la izquierda]
\\

\begin{Representacion}
\subsubsection{Justificación}
	\begin{Estructura}{Coordenada}[casillero]
		\begin{Tupla}[casillero]
			\tupItem{latitud}{Nat}
			\tupItem{longitud}{Nat}
		\end{Tupla}
	\end{Estructura}
	
\subsubsection{Invariante de representación}

\textbf{Informal}\\

Vale para todo par de natruales

\textbf{Formal}\\

\Rep[casillero e][e]{true}

\subsubsection{Predicado de abtraccion}

\AbsFc[casillero]{Coordenada}[e]{($\forall$ s:casillero)(Abs(s) $\igobs$ c:Coordenada) | (s.latitud = latitud(c) $\wedge$ s.longitud = longitud(c))}

\end{Representacion}

\subsection{Algoritmos}
\begin{Algoritmos}

\begin{algorithm}[H]{\textbf{iCrearCoordenada}(\In {n}{nat}, \In {m}{nat}) $\to$ res: casillero}
	\begin{algorithmic}[1]
		\State $res \gets <n, m>$ \Comment O(1)
		
		\medskip
		\Statex \underline{Complejidad:} O(1)
			\Statex \underline{Justificacion:} Sólo realiza una asignación
	\end{algorithmic}
\end{algorithm}

\begin{algorithm}[H]{\textbf{iDistEuclidea}(\In {c_1}{casillero}, \In {c_2}{casillero}) $\to$ res: nat}
	\begin{algorithmic}[1]
		\State $res \gets ((c_1.latitud - c_2.latitud)^{2} + (c_1.longitud  - c_2.longitud)^{2}))$ \Comment O(1)
				\medskip
		\Statex \underline{Complejidad:} O(1)
			\Statex \underline{Justificacion:} Sólo realiza operaciones básicas
	\end{algorithmic}
\end{algorithm}

\begin{algorithm}[H]{\textbf{iCoordenadaArriba}(\In {c}{casillero}) $\to$ res: casillero}
	\begin{algorithmic}[1]
		\State $res \gets <c.latitud +1, c.longitud>$ \Comment O(1)
		
		\medskip
		\Statex \underline{Complejidad:} O(1)
			\Statex \underline{Justificacion:} Sólo realiza una asignación y una suma
	\end{algorithmic}
\end{algorithm}

\begin{algorithm}[H]{\textbf{iCoordenadaAbajo}(\In {c}{casillero}) $\to$ res: casillero}
	\begin{algorithmic}[1]
		\State $res \gets <c.latitud -1, c.longitud>$ \Comment O(1)
		
		\medskip
		\Statex \underline{Complejidad:} O(1)
			\Statex \underline{Justificacion:} Sólo realiza una asignación y una resta
	\end{algorithmic}
\end{algorithm}

\begin{algorithm}[H]{\textbf{iCoordenadaIzquierda}(\In {c}{casillero}) $\to$ res: casillero}
	\begin{algorithmic}[1]
		\State $res \gets <c.latitud, c.longitud - 1>$ \Comment O(1)
		
		\medskip
		\Statex \underline{Complejidad:} O(1)
			\Statex \underline{Justificacion:} Sólo realiza una asignación y una resta
	\end{algorithmic}
\end{algorithm}


\begin{algorithm}[H]{\textbf{iCoordenadaDerecha}(\In {c}{casillero}) $\to$ res: casillero}
	\begin{algorithmic}[1]
		\State $res \gets <c.latitud, c.longitud + 1>$ \Comment O(1)
		
		\medskip
		\Statex \underline{Complejidad:} O(1)
			\Statex \underline{Justificacion:} Sólo realiza una asignación y una suma
	\end{algorithmic}
\end{algorithm}

\begin{algorithm}[H]{\textbf{iTieneCoordenadaAbajo}(\In {c}{casillero}) $\to$ res: bool}
	\begin{algorithmic}[1]
		\State $res \gets c.latitud > 0$ \Comment O(1)
		
		\medskip
		\Statex \underline{Complejidad:} O(1)
			\Statex \underline{Justificacion:} Si latitud es mayor a 0 tiene coordenada abajo
	\end{algorithmic}
\end{algorithm}

\begin{algorithm}[H]{\textbf{iTieneCoordenadaIzquierda}(\In {c}{casillero}) $\to$ res: bool}
	\begin{algorithmic}[1]
		\State $res \gets c.longitud > 0$ \Comment O(1)
		
		\medskip
		\Statex \underline{Complejidad:} O(1)
			\Statex \underline{Justificacion:} Si longitud es mayor a 0 tiene coordenada izquierda
	\end{algorithmic}
\end{algorithm}

\end{Algoritmos}
\newpage

\section{Mapa}


\subsection{Interfaz}

\parbox {1,7cm}{{\bf Género}} mapa\\
{\bf se explica con:}  \tadNombre{Conj($\sigma$)}, \tadNombre{bool}, \tadNombre{Coordenada}\\
\medskip

\InterfazFuncion{CrearMapa}{}{mapa}
{res = crearMapa()}
[O(1)]
[Crea un nuevo mapa]
\\

\InterfazFuncion{Coordenadas}{\In {m}{mapa}}{Conj(Coordenada)}
{res = coordenadas(m)}
[O(1)]
[Devuelve todas las coordenadas del mapa]
\\

\InterfazFuncion{AgregarCoordenada}{\In {c}{Coordenada}, \Inout {m}{mapa}}{}
[m $\igobs$ m$_0$]
{m = agregarCoor(c, m$_0$)}
[O(max($n^3$, $T^2$))]
[Agrega la coordenada a mapa. Donde T es el tama\~no de la grilla de todo el mapa (ancho * alto) y n es el cardinal de coordenadas en el Mapa. Donde n representa la cantidad de coordenadas en el mapa]
\\

\InterfazFuncion{HayCamino}{\In {c_1}{Coordenada}, \In {c_2}{Coordenada}, \In {m}{mapa}}{Bool}
[$c_1 \in coordenadas(m) \wedge c_2 \in coordenadas(m)$]
{res $\igobs$ hayCamino($c_1, c_2$)}
[O(1)]
[Te dice si dos coordenadas estan conectadas]
\\


\InterfazFuncion{PosExistente}{\In {c}{Coordenada},\In {m}{mapa}}{Bool}
{res $\igobs$ posExistente(c, m)}
[O(1)]
[Devuelve true si existe esa coordenada]
\\

\begin{Representacion}
\subsubsection{Justificación}
infomapa representa el tad Mapa. La componente coordenadas representa todas las coordenadas que fueron agregadas al mapa.
relacionCoordenadas representa que coordenadas estan conectadas con cuales. Es una matriz donde la primer dimension representa cada coordenada y la segunda dimensi\'on representa las coordenadas con las que est\'an relacionadas. La idea es guardar en la intercepci\'on entre dos coordenadas (cada una en una dimensi\'on distinta) un True en caso si están relacionadas. Esta matriz nos permite verificar en O(1) si dos coordenadas estan relacionadas o no.
ancho y alto representan el tama\~no del mapa, para calcularlo buscamos la coordenada con la longitud/latitud mayor y lo guardamos. Fue necesario guardar estos datos ya que los necesitamos para calcular la posici\'on de cada coordenada en relacionCoordenadas en cada dimensi\'on.

	\begin{Estructura}{Mapa}[infomapa]
		\begin{Tupla}[infomapa]
			\tupItem{coordenadas}{conj(Coordenada)}\\
			\tupItem{relacionCoordenadas}{arreglo(arreglo(Bool))}
			\tupItem{ancho}{Nat}
			\tupItem{alto}{Nat}
		\end{Tupla}
	\end{Estructura}

\subsubsection{Invariante de representación}

\textbf{Informal}\\
(1)Existe una coordenada en coordenadas tal que, dicha coordenada toma como su latitud al alto, y Existe una coordenada en coordenadas que su longitud es el ancho del mapa (puede ser la misma coordenada)\\
(2)relacionCoordenadas tiene como dimension el ancho*alto de alto y tambien de largo, ademas para toda celda de esta matriz se tiene la relacion de camino entre 2 coordenadas, dichas coordenadas tienen que existir en coordenadas del mapa 

\textbf{Formal}\\
\Rep[infoMapa $\quad$ M][M]{(1)($\exists$ $c_1$ :Coordenada)($c_1$ $\in$ M.coordenadas) $\impluego$ latitud($c_1$) = M.alto $\wedge$ ($\exists$ $c_2$ :Coordenada)($c_2$ $\in$ M.coordenadas) $\impluego$ longitud($c_2$) = M.ancho $\wedge$ (2) Tam(M.relacionCoordenadas) = M.ancho*M.alto $\wedge$ ($\forall$ i, j: nat)(Definido?(M.relacionCoordenadas, [i, j])) $\impluego$ <i, j> $\in$ M.coordenadas}


\subsubsection{Predicado de abtraccion}

\AbsFc[infoMapa]{Mapa}[M]{ mapa :Mapa | M.coordenadas = Coordenadas(mapa)}

\end{Representacion}

\subsection{Algoritmos}


\ttitulo{No exportable, operaciones auxiliares}

\InterfazFuncion{CoordenadasConectadasA}{\In {c}{Coordenada}, \In {m}{mapa}}{Conj(Coordenada)}
[$c \in coordenadas(m)$]
{$(\forall c_1: Coordenada) c_1 \in res \wedge c_1 \in coordenadas(m) \impluego hayCamino(c, c_1, m)$}
[O(n$^2$)]
[Devuelve un conjunto de coordenadas a las cuales hayCamino]
\\

\begin{Algoritmos}

\begin{algorithm}[H]{\textbf{iCrearMapa}() $\to$ res: mapainfo}
	\begin{algorithmic}[1]
		\State $res \gets <Vacio(), arreglo[0], 0, 0>$ \Comment O(1)

		\medskip
		\Statex \underline{Complejidad:} O(1)
			\Statex \underline{Justificacion:} Sólo realiza una asignación y las funciones de Vac\'io() de m\'odulo Conjunto Lineal y Diccionario Lineal son O(1)
	\end{algorithmic}
\end{algorithm}

\begin{algorithm}[H]{\textbf{iHayCamino}(\In {c1}{coordenada}, \In {c2}{coordenada}, \In {m} {infomapa}) $\to$ res: bool}
	\begin{algorithmic}[1]
		\State pos1 $\gets$ m.ancho * Longitud(c1) + m.alto * Latitud(c1) \Comment O(1)
		\State pos2 $\gets$ m.ancho * Longitud(c2) + m.alto * Latitud(c2) \Comment O(1)
		\State res $\gets$ m.relacionCoordenadas[pos1][pos2] \Comment O(1)

		\medskip
		\Statex \underline{Complejidad:} O(1)
			\Statex \underline{Justificacion:} Son solamente 3 asignaciones y un acceso de orden de 1 en un arreglo estatico
	\end{algorithmic}
\end{algorithm}

\begin{algorithm}[H]{\textbf{iPosExistente}(\In {c}{coordenada}, \In {m} {infomapa}) $\to$ res: bool}
	\begin{algorithmic}[1]
		\State res $\gets$ False \Comment O(1)
		\If {Latitud(c) $<$ m.alto $\wedge$ Longitud(c) $<$ m.ancho} \Comment O(1)
		\State pos $\gets$ m.ancho * Longitud(c) + m.alto * Latitud(c) \Comment O(1)
		\State res $\gets$ m.relacionCoordenadas[pos][pos] == True \Comment O(1)
		\Else
		\EndIf
		\medskip
		\Statex \underline{Complejidad:} O(1)
			\Statex \underline{Justificacion:} Son 3 asignaciones y un acceso de orden 1 a un arreglo. Esto funciona porque cuando calculo las relaciones entre las coordenas siempre definimos que una coordenada esta relacionada consigo misma.
	\end{algorithmic}
\end{algorithm}

\begin{algorithm}[H]{\textbf{iAgregarCoordenada}(\In {c}{coordenada}, \In {m}{infomapa})}
	\begin{algorithmic}[1]
		\State \IF Longitud(c) > infomapa.ancho THEN m.ancho $\gets$ Longitud(c) ELSE FI \Comment O(1)
		\State \IF Latitud(c) > infomapa.alto THEN m.alto $\gets$ Latitud(c) ELSE FI \Comment O(1)
		\State Agregar(m.coordenadas, c) \Comment O($\#$m.coordenadas)
		\State m.relacionCoordenadas $\gets$ arreglo[m.ancho*m.alto] de arreglo[m.ancho*m.alto] de Bool \Comment O($(m.ancho*m.alto)^2$)


		\State iter $\gets$ CrearIt(m.coordenadas) \Comment O(1)
		
		\While{HaySiguiente(iter)} \Comment O($\#$m.coordenadas$^3$)
			\State coor $\gets$ Siguiente(iter) \Comment O(1)
			\State Avanzar(iter) \Comment O(1)
			\State conectadas $\gets$ iCoordenadasConectadas(coor, m) \Comment O($\#$m.coordenadas$^2$)
			\State iterConectadas $\gets$ CrearIt(conectadas) \Comment O(1)
			\While {HaySiguiente(iterConectadas)} \Comment O($\#$m.coordenadas)
				\State $coor_2$ $\gets$ Siguiente(iterConectadas) \Comment(1)
				\State Avanzar(iterConectadas) \Comment O(1)
				\State pos1 $\gets$ m.ancho * Longitud(coor) + m.alto * Altitud(coor) \Comment O(1)
				\State pos2 $\gets$ m.ancho * Longitud($coor_2$) + m.alto * Altitud($coor_2$) \Comment O(1)
				\State m.relacionCoordenadas[pos1][pos2] $\gets$ True \Comment O(1)
				\State m.relacionCoordenadas[pos2][pos1] $\gets$ True \Comment O(1)
			\EndWhile
		\EndWhile
		
		\medskip
		\Statex \underline{Complejidad:} O(max($n^3$, $T^2$))
		\Statex \underline{Justificacion:} Donde T es el tama\~no de la grilla de todo el mapa (ancho * alto) y n es el cardinal de coordenadas en el Mapa. Ya que la creaci\'on de los arreglos no es gratis, tiene un costo que es el tama\~no del ancho*alto del Mapa. Tambi\'en ejecutamos un While de n iteraciones donde ejecutamos operaciones que cuestan como m\'aximo $n^2$ por lo cual el While tiene un costo del orden de $n^3$. Dado que la creaci\'on podr\'ia tomar m\'as tiempo que ejecutar el While debemos tomar el m\'aximo valor de ambos como la complejidad del algoritmo.
	\end{algorithmic}
\end{algorithm}

\begin{algorithm}[H]{\textbf{iCoordenadasConectadasA}(\In {c}{coordenada}, \In {m}{infomapa}) $\to$ res: Conj(coordenada)}
	\begin{algorithmic}[1]
		\State visitadas $\gets$ Vac\'io() \Comment O(1)
		\State aVisitar $\gets$ Encolar(Vac\'ia(), c) \Comment  O(1)
		\State res $\gets$ Agregar(Vac\'io(), c) \Comment O(1)
		\While{$\neg$ EsVac\'ia(aVisitar)} \Comment O($\#$m.coordenadas$^2$)
			\State coor $\gets$ Proximo(aVisitar) \Comment O(1)
			\State Desencolar(aVisitar)
			\State Agregar(visitadas, coor) \Comment O($\#$m.coordenadas)			
			\If {Latitud(coor) > 0} \Comment O(1)
					\State coorAbajo $\gets$ CoordenadaAbajo(coor) \Comment O(1)
					\If {$\neg$ Pertenece?(visitadas, coorAbajo) $\wedge$ Pertenece?(m.coordenadas, coorAbajo)} \Comment O($\#$m.coordenadas)
						\State Agregar(res, coorAbajo) \Comment O($\#$m.coordenadas)
						\State Encolar(aVisitar, coorAbajo) \Comment O(copy(coordenada))
					\Else
					\EndIf
			\Else
			\EndIf
			\If {longitud(coor) > 0} \Comment O(1)
					\State coorIzq $\gets$ CoordenadaIzquierda(coor) \Comment O(1)
					\If {$\neg$ Pertenece?(visitadas, coorIzq) $\wedge$ Pertenece?(m.coordenadas, coorIzq)} \Comment O($\#$m.coordenadas)
						\State Agregar(res, coorIzq) \Comment O($\#$m.coordenadas)
						\State Encolar(aVisitar, coorIzq) \Comment O(copy(coordenada))
					\Else
					\EndIf
			\Else
			\EndIf
			\State coorDer $\gets$ CoordenadaDerecha(coor) \Comment O(1)
			\If {$\neg$ Pertenece?(visitadas, coorDer) $\wedge$ Pertenece?(m.coordenadas, coorDer)} \Comment O($\#$m.coordenadas)
				\State Agregar(res, coorDer) \Comment O($\#$m.coordenadas)
				\State Encolar(aVisitar, coorDer) \Comment O(copy(coordenada))
			\Else
			\EndIf
			\State coorArriba $\gets$ CoordenadaDerecha(coor) \Comment O(1)
			\If {$\neg$ Pertenece?(visitadas, coorArriba) $\wedge$ Pertenece?(m.coordenadas, coorArriba)} \Comment O($\#$m.coordenadas)
				\State Agregar(res, coorArriba) \Comment O($\#$m.coordenadas)
				\State Encolar(aVisitar, coorArriba) \Comment O(copy(coordenada))
			\Else
			\EndIf
		\EndWhile

		\medskip
		\Statex \underline{Complejidad:} O($n^2$)
			\Statex \underline{Justificacion:} Dado un mapa y una coordenada te devuelve todas las coordenadas conectadas a esa coordenada inicial. Tomando como \textbf{n} el cardinal de infomapa.coordenadas nos da O($n^2$).
	\end{algorithmic}
\end{algorithm}


\end{Algoritmos}

\newpage
\section{Juego}

\subsection{Interfaz}

\parbox {1,7cm}{{\bf Género}} juego \\
{\bf se explica con:}  \tadNombre{Mapa}, \tadNombre{Conjunto ($\sigma$)} , \tadNombre{Secuencia($\sigma$)}\\
\medskip

\parbox {1,5cm}{\bf{Operaciones básicas}}\\

\InterfazFuncion{CrearJuego}{\In {m}{mapa}}{juego}
{res $\igobs$ crearJuego(m)}
[O(1)]
[Creo un nuevo juego tomando un mapa]\\

\InterfazFuncion{AgregarPokemon}{\In {p}{pokemon}, \In {c}{coordenada}, \Inout {g}{juego}}{}
[puedoAgregarPokemon(c, j) $\wedge$ g $\igobs$ g$_0$]
{g = agregarPokemon(p, c, g$_0$)}
[O(??)]
[Agrego un pokemon al juego]\\

\InterfazFuncion{AgregarJugador}{\In {g}{juego}}{it(jugador)}
[g $\igobs$ g$_0$]
{g = agregarJugador(g$_0$)}
[O(J), Siendo J la cantidad de jugadores que fueron agregados al juego]
[Agrega un jugador al juego]\\

\InterfazFuncion{Conectarse}{\In {j}{jugador}, \In {c}{coordenada}, \Inout {g}{juego}}{}
[g $\igobs$ g$_0$ $\wedge$ j $\in$ jugadores(g) $\yluego$ $\neg$estaConectado(j, g) $\wedge$ posExistente(c, mapa(g))]
{g = conectarse(g$_0$)}
[O(log(EC)), siendo EC la máxima cantidad de jugadoes esperando capturar un pokémon]
[Conecta un jugador al juego, con todo lo que esto implica]\\

\InterfazFuncion{Desconectarse}{\In {j}{jugador}, \Inout{g}{juego}}{}
[g $\igobs$ g$_0$ $\wedge$ j $\in$ jugadores(g) $\yluego$ estaConectado(j, g)]
{g = desconectarse(j, g$_0$)}
[O(log(EC)), siendo EC la máxima cantidad de jugadoes esperando capturar un pokémon]
[Desconecta al jugador j del juego]\\

\InterfazFuncion{Moverse}{\In {j}{jugador}, \In {c}{coordenada}, \Inout {g}{juego}}{}
[g $\igobs$ g$_0$ $\wedge$ j $\in$ jugadores(g) $\yluego$ estaConectado(j, g) $\wedge$ posExistente(c, mapa(g))]
{g = moverse(j, c, g$_0$)}
[O(??)]
[Mueve un jugador en el mapa, verifica si hay una captura de pokémon, y para el jugador movido verifica si cometió alguna infracción]\\

%%%%%%%%%%%%%%%%%%%%%%%%%%%%%%%%
% OBSERVADORES BASICOS
%%%%%%%%%%%%%%%%%%%%%%%%%%%%%%%%

\InterfazFuncion{Mapa}{\In {g}{Juego}}{Mapa}
{res $\igobs$ mapa(g)}
[O(1)]
[Devuelve la instancia de mapa que tenemos guardada]
\\

\InterfazFuncion{Jugadores}{\In {g}{Juego}}{Conj(Jugador)}
{res $\igobs$ jugadores(g)}
[O(1)]
[Devuelve la instancia de mapa que tenemos guardada]
\\

\InterfazFuncion{EstaConectado}{\In {j}{Jugador}, \In {g}{Juego}}{Bool}
[j $\in$ jugadores(g)]
{res $\igobs$ estaConectado(g)}
[O(1)]
[Dice si un jugador esta conectado o no]
\\

\InterfazFuncion{Sanciones}{\In {j}{Jugador}, \In {g}{Juego}}{Nat}
[j $\in$ jugadores(g)]
{res $\igobs$ sanciones(g)}
[O(1)]
[La cantidad de sanciones que tiene un jugador]
\\

\InterfazFuncion{Posicion}{\In {j}{Jugador}, \In {g}{Juego}}{Coordenada}
[j $\in$ jugadores(g) $\impluego$ estaConectado(j, g)]
{res $\igobs$ posicion(j, g)}
[O(1)]
[Posici\'on actual del jugador cuando se encuentra conectado]
\\

\InterfazFuncion{Pokemons}{\In {j}{Jugador}, \In {g}{Juego}}{Iter(<Pokemon, Nat>)}
[j $\in$ jugadores(g)]
{res $\igobs$ pok\'emons(j, g)}
[O(1)]
[Devuelve un iterador <Pokemon, Nat>]
\\

\InterfazFuncion{Expulsados}{\In {g}{Juego}}{Conj(Jugador)}
{res $\igobs$ expulsados(g)}
[O(J)]
[Conjunto de jugadores expulsados del juego]
\\

\InterfazFuncion{PosConPokemons}{\In {g}{Juego}}{Conj(coor)}
{res $\igobs$ posConPok\'emons(g)}
[O(1)]
[Conjunto de coordenadas con Pok\'emones]
\\

\InterfazFuncion{PokemonEnPos}{\In {c}{Coordenada}, \In {g}{Juego}}{Pokemon}
[c $\in$ posConPokemons(g)]
{res $\igobs$ pokemonEnPos(c, g)}
[O(1)]
[Devolvemos el Pokemon en la Coordenada]
\\

\InterfazFuncion{CantMovimientosParaCaptura}{\In {c}{Coordenada}, \In {g}{Juego}}{Nat}
[c $\in$ posConPokemons(g)]
{res $\igobs$ cantMovimientosParaCaptura(c, g)}
[O(1)]
[Cantidad de movimientos restantes para que un Pokemon sea capturado]
\\




%%%%%%%%%%%%%%%%%%%%%%%%%%%%%%%%
%
%%%%%%%%%%%%%%%%%%%%%%%%%%%%%%%%


\bf{Operaciones del iterador}\\

\InterfazFuncion{CrearIt}{\In {g}{game}}{itJugador}
{res = CrearItUni(v)}
[O(1)]
[Crea un iterador unidireccional no modificable al principio del vector, no necesariamente es un elemento válido por ende no se puede usar Actual sin Avanzar]
[]\\

\InterfazFuncion{HayMas?}{\In {it}{itJugador}}{bool}
{res $\igobs$ HayMas?(it)}
[O(n), siendo n la cantidad de elementos del vector]
[Devuelve true si y solo si quedan elementos para avanzar]\\

\InterfazFuncion{Avanzar}{\Inout {it}{itJugador}}{}
[HayMas?(it) $\wedge$ it $\igobs$ it$_0$]
{it $\igobs$ Avanzar(it$_0$)}
[O(n), siendo n la cantidad de elementos del vector]
[Avanza el iterador al próximo elemento del vector]\\

\InterfazFuncion{Actual}{\In {it}{itJugador}}{Nat}
[HayMas?(it)]
{res $\igobs$ Actual(it)}
[O(1)]
[Devuelve el Id apuntado por el iterador]
[res no es modificable porque el iterador no es modificable]\\

\InterfazFuncion{Siguientes}{\In {it}{itJugador)}}{lista($\sigma$)}
{res $\igobs$ Siguientes(it)}
[O(n), siendo n la cantidad de elementos del vector]
[Devuelve los elementos del vector posteriores al iterador, puede no haber ninguno]\\

\bf{No exportable, operaciones auxiliares}\\

\InterfazFuncion{CeldasValidas}{\In {g}{juego}, \In {c}{coordenada}}{lista(coordenadas)}
[c $\in$ coordenadas(mapa(g))]
{($\forall c_1$: coordenada)(esta?($c_1$, res)$\impluego$(distEuclidea($c_1$, c)$\leq$2  $\wedge$ posExistente($c_1$, mapa(g)))}
[Devuelve una lista con las coordenas a una distancia no mayor de 2 de la coordenada c y que ademas existan en el mapa del juego]

\begin{Representacion}
\Titulo{Representacion del modulo}
\subsubsection{Justificación}
Game representa un Juego.\\
En pokemons guardamos un diccionario sobre Trie y la clave es el Pokemon (string). Esto nos permite encontrar un pokemon en |P|. En el diccionario guardamos la cantidad de pokemones salvajes que hay y cuantos hubieron de ese tipo en total para poder calcular la rareza del pokemon en O(|P|) junto con cantidadTotPokemons que es el total de los Pokemons en el juego.
En mapa guardamos el mapa con el que se crea el Juego.\\
coordenadasConPokemons guardamos un conjunto de Coordenadas donde hay pokemones para poder devolver en posConPok\'emons en O(1).
En jugadores de game guardamos un vector de InfoJugador (ver m\'as abajo). Como en InfoCoordenada guardamos una lista de jugadores en cada coordenada y cuando expulsamos un jugador debemos borrar de forma eficiente en esa lista guardamos un iterador al jugador en diccionario para poder cumplir con los ordenes de mover.\\
mapainfo es una matriz de InfoCoordenadas donde cada dimensi\'on es la latitud y longitud de una coordenada.\\
cantidadTotPokemones guardamos el total de pokemones del juego.\\
coordenaadasPokemons es un diccionario $<$Coordenada, Pokemon$>$ donde guardamos el Pokemon que cada coordenada del Mapa tiene.\\
pokemonsDeJugadores es una lista donde guardamos los Pokemones que atrapó cada jugador. (\textbf{**}) Esto lo tuvimos que guardar fuera de InfoJugador porque ten\'iamos un problema de complejidad al agregar un jugador nuevo. Como usabamos un Vector de jugadores el agregar es O(J + copy($\alpha$)) y guardar los Pokemones que atrapo el jugador hacía que el copy no fuera O(1), entonces guardamos un iterador a esta lista en InfoJugador.\\
InfoCoordenada representa la informaci\'on perteneciente a cada coordenada en el Mapa. Guardamos si en esta posici\'on hay pokemon y que pokemon, si ya se capturo un pokemon en esta coordenada que lo usamos para saber si quedo un heap no valido. En jugEspe usamos un min heap de $<$Jugador, CantidadPokemonesAtrapados$>$ para cuando se capture un Pokemon poder seleccionar el jugador que lo atrapa en O(log n) porque lo desencolamos del heap. Era necesario esta complejidad para poder cumplir la complejidad de mover. En MovimientosRestantes guardamos la cantidad de movimientos restantes para atrapar el Pokemon.\\
InfoJugador representa la informaci\'on de cada jugador. Aqu\'i guardamos al jugador (para no depender solamente del la posici\'on del vector que algunas veces lo recorremos pero no lo tenemos), si esta conectado, expulsado, sanciones, en pos la posici\'on actual, pokemons atrapados, la posici\'on en el mapa, cantTotalPoke es cantidad total de pokemones que atrapo. En pokemons guardamos un iterador  de lista que se guarda en Game, se explic\'o en (\textbf{**}). En posicionMapa guardamos un iterador a la cola de prioridad (heap) de la coordenada en caso de estar esperando para atrapar un pokemon, esto lo hicimos para poder eliminarlo de forma f\'acil del heap (cuando se mueve, se elimina, etc) y poder cumplir las complejidades de mover.


	\begin{Estructura}{Juego}[Game]
		\begin{Tupla}[Game]
			\tupItem{pokemons}{diccString(pokemon, tupla <cantSalvaje: Nat, cantTotal: Nat>)}
			\tupItem{\\mapa}{Mapa}
			\tupItem{\\jugadores}{Vector(InfoJugador)}
			\tupItem{\\mapaInfo}{Arreglo de Arreglo de InfoCoordenada}
			\tupItem{\\cantidadTotPokémons}{Nat}
			\tupItem{\\coordenadasConPokemons}{Conj(Coordenada)}
			\tupItem{\\coordenadasPokemons}{Dicc(Coordenada, pokemon)}
			\tupItem{\\pokemonsDeJugadores}{Lista(Dicc(pokemon: string, cant: nat))}
		\end{Tupla}
	\end{Estructura}
	
	\begin{Tupla}[InfoJugador]
		\tupItem{jug}{jugador}
		\tupItem{\\conectado}{Bool}
		\tupItem{\\expulsado}{Bool}
		\tupItem{\\sanciones}{Nat}
		\tupItem{\\pos}{Coordenada}
		\tupItem{\\pokemons}{itLista(Dicc(pokemon: string, cant: nat))}
		\tupItem{\\posicionMapa}{itDicc(jugador: Nat, EsperandoCapturar: itColaDePrioridad(cantPokemon)))}
		\tupItem{\\cantTotalPoke}{Nat}
	\end{Tupla}
	
	
	\begin{Tupla}[InfoCoordenada]
		\tupItem{pokemon}{Pokemon}
		\tupItem{\\jugEspe}{ColaDePrioridad(cantPokemon)}
		\tupItem{\\hayPokemon}{Bool}
		\tupItem{\\yaSeCapturo}{Bool}
		\tupItem{\\jugadores}{Dicc(jugador: nat, EsperandoCapturar: itColaDePrioridad(cantPokemon))}
		\tupItem{\\MovimientosRestantes}{Nat}
	\end{Tupla}

\subsubsection{Invariante de representación}

\subsubsection{Predicado de abtraccion}

	\Titulo{Representacion del iterador}
\subsection{Justificacion}

	\begin{Estructura}{itJugador}[iter]
		\begin{Tupla}[iter]
		\tupItem{posicion}{nat}
		\tupItem{vector}{puntero(Vector(infoJugador))}
		\end{Tupla}
	\end{Estructura}
\subsection{Invariante de representacion}

\textbf{Informal}

\textbf{Formal}

\subsection{Predicado de abstraccion}

\end{Representacion}
\subsection{Algoritmos}

\begin{Algoritmos}
\Titulo {Algoritmos del Modulo}


%%%%%%%%%%%%%%%%%%%%%%%%%%%%%%%%
% Generadores
%%%%%%%%%%%%%%%%%%%%%%%%%%%%%%%%

\begin{algorithm}[H]{\textbf{iCrearJuego}(\In {m}{Mapa}) $\to$ res: Game}
	\begin{algorithmic}[1]
		\State coords $\gets$ Coordenadas(m) \Comment O(1)
		\State iter $\gets$ CrearIt(coords) \Comment O(1)
		\State ancho $\gets$ 0 \Comment O(1)
		\State alto $\gets$ 0 \Comment O(1)
		\While{HaySiguiente(iter)} \Comment O($\#$coords)
			\State c $\gets$ Siguiente(iter) \Comment O(1)
			\State Avanzar(iter) \Comment O(1)
			\State \IF Altitud(c) > alto THEN alto $\gets$ Altitud(c) ELSE FI \Comment O(1)
			\State \IF Longitud(c) > ancho THEN ancho $\gets$ Longitud(c) ELSE FI \Comment O(1)
		\EndWhile
		\State infocoor $\gets$ arreglo[ancho] de arreglo[alto] de <Vac\'ia(), Bool, Vac\'ia(), 10> \Comment O(m.ancho * m.alto)
		\State res $\gets$ <Vacio(), m, Vac\'ia(), infocoor, 0, Vac\'ia()> \Comment O(1)
		\medskip
		\Statex \underline{Complejidad:} O(TM)
		\Statex \underline{Justificacion:} Donde TM es el tama\~no del mapa (alto $\times$ ancho)
	\end{algorithmic}
\end{algorithm}


\begin{algorithm}[H]{\textbf{iAgregarJugador}(\In {g}{Game}) $\to$ res: itJuego($\sigma$)}
	\begin{algorithmic}[1]
		\State Dicc(pokemon, nat) dicc $\gets$ Vacio() \Comment O(1)
		\State itLista(Dicc(pokemon, cantidad)) it $\gets$ AgregarAtras(g.pokemonsDeJugadores, dicc) \Comment O(1) 
		\State AgregarAtras (g.jugadroes, <false, false, 0, <0, 0>, it, NULL, 0>) \Comment O(longitud(g.jugadores) + copy(tupla))
		\State   itJuego($\sigma$) it $\gets$ CrearIt(g) \Comment O(1)
		\While {it.posicion $<$ Longitud(g.jugadores)} \Comment O(longitud(g.jugadores))
			\State it.posicion $\gets$ it.posicion + 1 \Comment O(1)
		\EndWhile
		\State res $\gets$ it \Comment O(1)
		\medskip
		\Statex \underline{Complejidad:} O(longitud(g.jugadores))
		\Statex \underline{Justificacion:}Agrega un jugador al juego, el costo de copiar la tupla es O(1) porque todas las componentes están vacías, después crea un iterador al principio del vector y lo avanza hasta la última posición donde fue agregado el jugador y lo devuelve. Para hacer esto último tengo que recorrer todo el vector entonces la complejidad final es O(longitud(g.jugadores) + (longitud(g.jugadores) + copiar(tupla))), como dijimos el costo de copiar  
	\end{algorithmic}
\end{algorithm}

\begin{algorithm}[H]{\textbf{iAgregarPokemon}(\In {p}{string}, \In {c}{coordenada}, \Inout {g}{game})}
	\begin{algorithmic}[1]
		\State AgregarRapido(g.coordenadasPokemons, c) \Comment O(copiar(c)) = O(1)
		\If{$\neg$Definido(p, g.pokemons)} \Comment O(2$*$|p|) = O(|p|)
			\State Definir(p, <1, 1>, g.pokemons) \Comment O(|P|)
		\Else
			\State Definir(p, <Significado(p, g.pokemons).cantSalvaje +1, Significado(p, g.pokemons).cantTotales +1>, g.pokemons) \Comment O(|p|)
		\EndIf
		\State g.mapaInfo[c].hayPokemon $\gets$ true \Comment O(1)
		\State g.mapaInfo[c].jugEspe $\gets$ Vacio() \Comment O(1)
		\State g.mapaInfo[c].yaSeCapturo $\gets$ flase \Comment O(1)
 		\State g.mapaInfo[c].movimientosRestantes $\gets$ 0 \Comment O(1)
		\State lista(coordenada) lc $\gets$ Vacia() \Comment O(1)
		\State lc $\gets$ CeldasValidas(g, c) \Comment O(1)
		\State AgregarAtras(lc, c) \Comment O(1)
		\State itLista(coordenadas) itCoordenadas $\gets$ CrearIt(lc) \Comment O(1)
		\While{HaySiguiente(itCoordenadas)} \Comment O($\#$(jugadoresEnRadioDeCaptura)*log($\#$(jugadoresEnRadioDeCaptura))
			\State itDicc(jugador, EsperandoCapturar) itJugadores $\gets$ CrearIt(g.mapaInfo[Siguiente(itCoordenadas)].jugadores) \Comment O(1)
			\While {HaySiguiente(itJugadores)} \Comment O($\#$(jugadoresEnRadioDeCaptura)*log($\#$(jugadoresEnRadioDeCaptura))
				\If{SiguienteSignificado(itJugadores) $\neq$ NULL}
					\State Borrar(SiguienteSignificado(itJugadores))				
				\EndIf
				\State itColaPrioridad itCola $\gets$ Encolar(g.mapaInfo[c].jugEspe, g.jugadores[SiguienteClave(itJugadores)].cantTotPoke, SiguienteClave(itJugadores)) \Comment O(log (n), siendo n la cantidad de elementos en el arbol)
				\State SiguienteSignficado(itJugadores) $\gets$ itCola \Comment O(1)
				\State Avanzar(itJugadores) \Comment O(1)
			\EndWhile
			\If {EsVacio?(g.mapaInfo[Siguiente(itCoordena)].jugEspe)}
				\State g.mapaInfo[Siguiente(itCoordena)].yaSeCapturo $\gets$ false \Comment O(1)			
			\EndIf
			\State Avanzar(itCoordenada) \Comment O(1)
		\EndWhile
		\medskip
		\Statex \underline{Complejidad:} O(($\#$(jugadoresEnRadioDeCaptura)*log($\#$(jugadoresEnRadioDeCaptura))) + |P|)
		\Statex \underline{Justificacion:}Primero defino el pokemon en el diccString, si ya estaba sumo un 1 en la cant de pokemons salvajes, sino lo defino con un 1 en cant salvajes y 0 en atrapados, esto me toma |P| siendo P la máxima longitud de una clave del diccionario. Después me armo una lista con las coordenadas en el radio de captura esto toma tiempo constate porque son finitas cordenadas. Luego creo un iterador a esta lista para recorrerla, como tengo finitos elementos recorrerla es constante. Por cada coordenada me creo un it al diccionario de los jugadores en esa coordenada. Los recorro, recorrer a todos los jugadores en el radio es EC siendo EC la máxima cantidad de jugadores esperando capturar un pokemon, y por cada jugador pregunto si su iter a cola de prioridad esta definido (no supimos como hacerlo asi que lo consideramos como un puntero) si lo esta borra lo que esta apuntando, despues encolo el elemento a la cola de prioridad del pokemon, me guardo el iterador que me devuelve encolar, y asi para todos los jugadores de la coordena, despues de salir de este while que toma la cantidad de los jugadores de la coordenada, O($\#$jugadores en la coordenada), y por cada uno lo encolo a la cola eso me toma O(log n)(n la cantidad de elementos de la cola), entonces la complejidad final es O($\#$jugadores en la coordenada * log n). Esto lo hago para todas las coordenadas entonces me queda O($\#$JugadoresEsperandoCapturar * log($\#$JugadoresEsperandoCapturar)), luego antes de avanzar de coordenada, pregunto si el la que estoy parado su cola de prioridad esta vacía, si lo está pongo a false un booleano, este es el caso de que hubo alguna vez un pokemon en esa coordenada. Y despues de salir del while general termine todo entonces la complejidad final es O(($\#$(jugadoresEnRadioDeCaptura)*log($\#$(jugadoresEnRadioDeCaptura))) + |P|) 
	\end{algorithmic}
\end{algorithm}

\begin{algorithm}[H]{\textbf{iCeldasValidas}(\In {g}{Game}, \In {c}{coordenada}) $\to$ res: lista(coordenada)}
	\begin{algorithmic}[1]
		\State lista(coordenada) ls $\gets$ Vacia() \Comment O(1)
		\State nat i $\gets$ 4 \Comment O(1)
		\While {i $>$ 0} \Comment O(1)
			\State AgregarAtras(ls, <latitud(c)+i, longitud(c)>) \Comment O(1)
			\State AgregarAtras(ls, <latitud(c)-i, longitud(c)>) \Comment O(1)
			\State AgregarAtras(ls, <latitud(c), longitud(c)+i>) \Comment O(1)
			\State AgregarAtras(ls, <latitud(c), longitud(c)-i>) \Comment O(1)
			\State i $\gets$ i - 1 \Comment O(1)
		\EndWhile
		\State i $\gets$ 3
		\While {i > 0} \Comment O(1)
			\State AgregarAtras(ls, <latitud(c)+3, longitud(c)-(i-1)>) \Comment O(1)
			\State AgregarAtras(ls, <latitud(c)-(i-1), longitud(c)+3>) \Comment O(1)
			\State AgregarAtras(ls, <latitud(c)-3, longitud(c)-(i-1)>) \Comment O(1)
			\State AgregarAtras(ls, <latitud(c)-(i-1), longitud(c)-3>) \Comment O(1)
			\State AgregarAtras(ls, <latitud(c)-3, longitud(c)+(i-1)>) \Comment O(1)
			\State AgregarAtras(ls, <latitud(c)+(i-1), longitud(c)-3>) \Comment O(1)
			\State AgregarAtras(ls, <latitud(c)+3, longitud(c)+(i-1)>) \Comment O(1)
			\State AgregarAtras(ls, <latitud(c)+(i-1), longitud(c)+3>) \Comment O(1)
			\State AgregarAtras(ls, <latitud(c)+(i-1), longitud(c)+2>) \Comment O(1)
			\State AgregarAtras(ls, <latitud(c)+(i-1), longitud(c)+1>) \Comment O(1)
			\State AgregarAtras(ls, <latitud(c)+(i-1), longitud(c)-2>) \Comment O(1)
			\State AgregarAtras(ls, <latitud(c)+(i-1), longitud(c)-1>) \Comment O(1)
			\State AgregarAtras(ls, <latitud(c)-(i-1), longitud(c)-2>) \Comment O(1)
			\State AgregarAtras(ls, <latitud(c)-(i-1), longitud(c)-1>) \Comment O(1)
			\State AgregarAtras(ls, <latitud(c)-(i-1), longitud(c)+2>) \Comment O(1)
			\State AgregarAtras(ls, <latitud(c)-(i-1), longitud(c)+1>) \Comment O(1)			
			\State i $\gets$ i -1 \Comment O(1) 
		\EndWhile
		\State itLista(coordenada) it $\gets$ CrearIt(ls) \Comment O(1)		
		\While {HaySiguiente(it)} \Comment O(1)
			\If{PosExistente (Siguiente(it), g.mapa)} \Comment O(1)
				\State Avanzar(it) \Comment O(1) 
			\Else
				\State EleminarSiguiente(it) \Comment O(1)			
			\EndIf
		\EndWhile
		\State res $\gets$ ls \Comment O(1)
		\medskip
		\Statex \underline{Complejidad:} O(1))
		\Statex \underline{Justificacion:}Como me estoy fijando un numero finito de coordenadas, y la cantidad que veo no varia porque no depende de la entrada, puedo decir que toma O(1) ver todas las celdas, luego recorro la lista para ver cuales son válidas y cuales no, que como son una cantidad constatnte de celdas recorrer la lista tambien es constante
	\end{algorithmic}
\end{algorithm}


\begin{algorithm}[H]{\textbf{iAgregarJugador}(\In {g}{Game}) $\to$ res: itJuego($\sigma$)}
	\begin{algorithmic}[1]
		\State Dicc(pokemon, nat) dicc $\gets$ Vacio() \Comment O(1)
		\State itLista(Dicc(pokemon, cantidad)) it $\gets$ AgregarAtras(g.pokemonsDeJugadores, dicc) \Comment O(1) 
		\State AgregarAtras (g.jugadroes, <false, false, 0, <0, 0>, it, NULL, 0>) \Comment O(longitud(g.jugadores) + copy(tupla))
		\State   itJuego($\sigma$) it $\gets$ CrearIt(g) \Comment O(1)
		\While {it.posicion $<$ Longitud(g.jugadores)} \Comment O(longitud(g.jugadores))
			\State it.posicion $\gets$ it.posicion + 1 \Comment O(1)
		\EndWhile
		\State res $\gets$ it \Comment O(1)
		\medskip
		\Statex \underline{Complejidad:} O(longitud(g.jugadores))
		\Statex \underline{Justificacion:}Agrega un jugador al juego, el costo de copiar la tupla es O(1) porque todas las componentes están vacías, después crea un iterador al principio del vector y lo avanza hasta la última posición donde fue agregado el jugador y lo devuelve. Para hacer esto último tengo que recorrer todo el vector entonces la complejidad final es O(longitud(g.jugadores) + (longitud(g.jugadores) + copiar(tupla))), como dijimos el costo de copiar  
	\end{algorithmic}
\end{algorithm}


\begin{algorithm}[H]{\textbf{iMoverse}(\In {j}{jugador}, \In {c}{coordenada}, \Inout {g}{Game})}
	\begin{algorithmic}[1]
		\If {SiguienteSignificado(g.jugadores[j].posicionMapa) $\neq$ NULL} \Comment O(1)
				\State Borrar(SiguienteSignificado(g.jugadores[j].posicionMapa)) \Comment O(1)
				\State g.jugadores[J].posicionMapa $=$ NULL \Comment O(1)
		\EndIf		
		\If{distEuclidia(c, g.jugadores[j].pos) $>$ 100 $\vee$ $\neg$(hayCamino(c,g.jugadores[j].pos, g.mapa))} \Comment O(1)
			\State g.jugadores[j].sanciones $\gets$ g.jugadores[j].sanciones +1 \Comment O(1)
			\If(g.jugadores[j].sanciones $\geq$ 4) \Comment O(1)
				\State g.jugadores[j].expulsado $\gets$ true \Comment O(1)			
			\EndIf
		\EndIf
		\If {g.jugadores[j].expulsado $=$ true}
			\State g.cantidadTotPokemons $\gets$ g.cantidadTotPokemons - g.jugadores[j].cantTotalPoke \Comment O(1)
			\State itDicc(pokemon, nat) itPokemons $\gets$ CrearIt(Siguiente(g.jugadores[j].pokemons))
			\While{HaySiguiente(itPokemons)} \Comment O($\#$(pokemonsDicc)*|P|)
				\State Significado(g.pokemones, SiguienteClave(itPokemons)).cantTotal - SiguienteSignificado(itPokemons) \Comment O(|P|)
				\If {Significado(g.pokemones, SiguienteClave(it.Pokemons)).cantTotal = 0}
					\State Borrar(g.pokemons, SiguienteClave(it.Pokemons)) \Comment O(|P|)					\EndIf				
				\State EliminarSiguiente(itPokemons) \Comment O(1)		
			\EndWhile
			\State EliminarSiguiente(g.jugadores[j].pokemons) \Comment O(1)
			\State g.jugadores[j].pokemons $\gets$ NULL \Comment O(1)
			\State g.jugadores[j].cantTotalPoke $\gets$ 0 \Comment O(1)
		\Else
			\State lista(coordenada) lc $\gets$ CeldasValidas(g.jugadores[j].pos, g.mapa) \Comment O(1)
			\State itLista(coordenada) itCoordenada $\gets$ lc \Comment O(1)
			\While {HaySiguiente(itCoordenada)} \Comment O(1)
				\If {g.mapaInfo[Siguiente(itCoordenada)].hayPokemon $=$ false} \Comment O(1)
					\State EleminarSiguiente(itCoordenada) \Comment O(1)
				\EndIf
			\EndWhile
			\State g.jugadores[j].pos $\gets$ c \Comment O(1)
			\If{itCoordenada $\neq$ NULL}
				\State itDicc(jugador, itColaPrioridad(cantidadPokemons)) itDicc $\gets$ DefinirRapido(g.mapaInfo[c].jugadores, j, Encolar(g.mapaInfo[Siguiente(itCoordenada)].jugEspe, g.jugadores[j].cantTotalPoke, j)) \Comment O(log ($\#$elemEnLaCola))
				\State g.jugadores[j].posicionMapa $\gets$ itDicc \Comment O(1)
			\Else
				\State itDicc(jugador, itColaPrioridad(cantidadPokemons)) itDicc $\gets$ DefinirRapido(g.mapaInfo[c].jugadores, j, NULL)	\Comment O(1)		
				\State g.jugadores[j].posicionMapa $\gets$ itDicc
			\EndIf
		\EndIf
		\State itDicc(coordenada, pokemons) itPokemons $\gets$ g.coordenadasPokemons \Comment O(1)
		
		\medskip
		\Statex \underline{Complejidad:} 
		\Statex \underline{Justificacion:}
	\end{algorithmic}
\end{algorithm}

%%%%%%%%%%%%%%%%%%%%%%%%%%%%%%%%
% OBSERVADORES BASICOS
%%%%%%%%%%%%%%%%%%%%%%%%%%%%%%%%

\begin{algorithm}[H]{\textbf{iMapa}(\In {g}{Game}) $\to$ res: mapa}
	\begin{algorithmic}[1]
		\State res $\gets$ g.mapa \Comment O(1)
		\medskip
		\Statex \underline{Complejidad:} O(1)
		\Statex \underline{Justificacion:} Devuelve la instancia de mapa que tenemos guardada
	\end{algorithmic}
\end{algorithm}

\begin{algorithm}[H]{\textbf{iJugadores}(\In {g}{Game}) $\to$ res: conj(jugador)}
	\begin{algorithmic}[1]
		\State res $\gets$ CrearIt(g) \Comment O(1)
		\medskip
		\Statex \underline{Complejidad:} O(1)
		\Statex \underline{Justificacion:} Crea un iterador a jugadores que tiene orden de 1 y lo devuelve
	\end{algorithmic}
\end{algorithm}

\begin{algorithm}[H]{\textbf{iEstaConectado}(\In {j}{jugador}, \In {g}{Game}) $\to$ res: conj(jugador)}
	\begin{algorithmic}[1]
		\State res $\gets$ g.jugadores[j].conectado \Comment O(1)
		\medskip
		\Statex \underline{Complejidad:} O(1)
		\Statex \underline{Justificacion:} Es una asignaci\'on, un acceso de O(1) a un vector por el id del jugador y ah\'i guardamos una tupla con informaci\'on del jugador, en particular si est\'a \'o no conectado
	\end{algorithmic}
\end{algorithm}

\begin{algorithm}[H]{\textbf{iSanciones}(\In {j}{jugador}, \In {g}{Game}) $\to$ res: nat}
	\begin{algorithmic}[1]
		\State res $\gets$ g.jugadores[j].sanciones \Comment O(1)
		\medskip
		\Statex \underline{Complejidad:} O(1)
		\Statex \underline{Justificacion:} Es una asignaci\'on, un acceso de O(1) a un vector por el id del jugador y ah\'i guardamos una tupla con informaci\'on del jugador, en particular la cantidad de sanciones que tiene
	\end{algorithmic}
\end{algorithm}

\begin{algorithm}[H]{\textbf{iPosicion}(\In {j}{jugador}, \In {g}{Game}) $\to$ res: coordenada}
	\begin{algorithmic}[1]
		\State res $\gets$ g.jugadores[j].pos \Comment O(1)
		\medskip
		\Statex \underline{Complejidad:} O(1)
		\Statex \underline{Justificacion:} Es una asignaci\'on, un acceso de O(1) a un vector por el id del jugador y ah\'i guardamos una tupla con informaci\'on del jugador, en particular la posici\'on actual del jugador cuando esta conectado
	\end{algorithmic}
\end{algorithm}

\begin{algorithm}[H]{\textbf{iPokemons}(\In {j}{jugador}, \In {g}{Game}) $\to$ res: itDicc($<Pokemon, cantidad>$)}
	\begin{algorithmic}[1]
		\State res $\gets$ CrearIt(Siguiente(g.jugadores[j].pokemons)) \Comment O(1)
		\medskip
		\Statex \underline{Complejidad:} O(1)
		\Statex \underline{Justificacion:} Dentro de jugadores guardamos un iterador a Dicc(<Pokemon, Cantidad>) que est\'a guardado en Game. De esta forma en el vector de jugadores guardamos estructuras simples de copiar, esto era necesario por la complejidad del agregar jugador (ver AgregarJugador). Tanto la asignaci\'on y la creaci\'on del iterador y el siguiente del iterador de lista son todos O(1)
	\end{algorithmic}
\end{algorithm}

\begin{algorithm}[H]{\textbf{iExpulsados}(\In {g}{Game}) $\to$ res: conj(jugador)}
	\begin{algorithmic}[1]
		\State res $\gets$ Vac\'io() \Comment O(1)
		\State tam $\gets$ Longitud(g.jugadores) \Comment O(1)
		\State \textbf{for} n $\gets$ 0 \textbf{to} tam \Comment O(J)
		\State \,\,\,\,AgregarRapido(res, g.jugadores[n].jug) \Comment O(copy(jugador))
		\medskip
		\Statex \underline{Complejidad:} O(J)
		\Statex \underline{Justificacion:} El for se ejecuta J veces y como jugador es un nat, el costo de copiarlo es O(1) entonces la complejidad es O(J)
	\end{algorithmic}
\end{algorithm}

\begin{algorithm}[H]{\textbf{iPosConPokemons}(\In {g}{Game}) $\to$ res: conj(coordenada)}
	\begin{algorithmic}[1]
		\State res $\gets$ g.coordenadasConPokemons \Comment O(1)
		\medskip
		\Statex \underline{Complejidad:} O(1)
		\Statex \underline{Justificacion:} Devolvemos el conjunto de coordenadas con Pokemones. La asignaci\'on al ser de un tipo \textbf{no} primitivo tanto res como g.coordenadasConPokemons referencian a la misma estructura f\'isica (del apunte de dise\~no)
	\end{algorithmic}
\end{algorithm}

\begin{algorithm}[H]{\textbf{iPokemonEnPos}(\In {c}{coordenada}, \In {g}{Game}) $\to$ res: pokemon}
	\begin{algorithmic}[1]
		\State res $\gets$ g.mapainfo[Altitud(c)][Longitud(c)].pokemon \Comment O(1)
		\medskip
		\Statex \underline{Complejidad:} O(1)
		\Statex \underline{Justificacion:} El acceso a mapainfo que es una matriz es O(1) y tenemos solo una asignaci\'on
	\end{algorithmic}
\end{algorithm}

\begin{algorithm}[H]{\textbf{iCantMovimientosParaCaptura}(\In {c}{coordenada}, \In {g}{Game}) $\to$ res: nat}
	\begin{algorithmic}[1]
		\State res $\gets$ 10 - g.mapainfo[Altitud(c)][Longitud(c)].MovimientosRestantes \Comment O(1)
		\medskip
		\Statex \underline{Complejidad:} O(1)
		\Statex \underline{Justificacion:} El acceso a mapainfo que es una matriz es O(1), tenemos solo una asignaci\'on y una resta
	\end{algorithmic}
\end{algorithm}






%%%%%%%%%%%%%%%%%%%%%%%%%%%%%%%%
% Iterador
%%%%%%%%%%%%%%%%%%%%%%%%%%%%%%%%

\Titulo {Algoritmos del iterador}

\begin{algorithm}[H]{\textbf{iCrearIt}(\In {g}{game}) $\to$ res: iter))}
	\begin{algorithmic}[1]
		\State res $\gets$ <0, puntero(g.jugadores)> \Comment O(1)
		\medskip
		\Statex \underline{Complejidad:} O(1)
		\Statex \underline{Justificacion:}Crea un iterador al princio del vector, solo 
		realiza una asignación  
	\end{algorithmic}
\end{algorithm}

\begin{algorithm}[H]{\textbf{iHayMas?}(\In {it}{iter}) $\to$ res: bool))}
	\begin{algorithmic}[1]
		\State bool b $\gets$ false
		\State nat i $\gets$ it.posicion \Comment O(1)
		\While {i $<$ it.vector$\to$ longitud $\wedge$ $\neg$b} \Comment O(longitud(v))
			\If {it.vector$\to$ elementos[i].expulsado $=$ false} \Comment O(1)
				\State b $\gets$ true \Comment O(1)
			\EndIf
		\State i $\gets$ i +1 \Comment O(1)		
		\EndWhile
		\State res $\gets$ b
		\medskip
		\Statex \underline{Complejidad:} O(longitud(v))
		\Statex \underline{Justificacion:}En el peor de lo casos hay que recorrer todo el vector para saber si existe otro elemento

\end{algorithmic}
\end{algorithm}	
		
\begin{algorithm}[H]{\textbf{iAvanzar}(\Inout {it}{iter})}
	\begin{algorithmic}[1]
		\State bool b $\gets$ false
		\State nat i $\gets$ it.posicion \Comment O(1)
		\While {i $<$ it.vector$\to$ longitud $\wedge$ $\neg$b} \Comment O(longitud(v))
			\If {it.vector$\to$ elementos[i].expulsado $=$ false} \Comment O(1)
				\State b $\gets$ true \Comment O(1)
			\EndIf
		\State i $\gets$ i +1 \Comment O(1)		
		\EndWhile
		\State it.posicion $\gets$ (i-1) \Comment O(1)
		\medskip
		\Statex \underline{Complejidad:} O(longitud(v))
		\Statex \underline{Justificacion:}Recorre el vector y para en el primer elemento válido, que existe por la precondición de la función, luego actualiza la posición del iterador  
	\end{algorithmic}
\end{algorithm}

\begin{algorithm}[H]{\textbf{iActual}(\In {it}{iter}) $\to$ res: nat}
	\begin{algorithmic}[1]
		\State res $\gets$ it.posicion \Comment O(1)
		\medskip
		\Statex \underline{Complejidad:} O(1)
		\Statex \underline{Justificacion:}Devuelve la id del jugador que apunta el iterador  
	\end{algorithmic}
\end{algorithm}
	
\begin{algorithm}[H]{\textbf{iSiguientes}(\In {it}{iter}) $\to$ res: lista(jugadores)))}
	\begin{algorithmic}[1]
		\State lista(nat) ls $\gets$ Vacia() \Comment O(1), crea una lista vacia
		\State nat i $\gets$ it.posicion \Comment O(1)
		\While {i $<$ it.vector$\to$ longitud} \Comment O(longitud(v))
			\If {it.vector$\to$ elementos[i].expulsado $=$ false} \Comment O(1)
				\State AgregarAtras(ls, it.posicion) \Comment O(1)
			\EndIf
		\State i $\gets$ i +1 \Comment O(1)		
		\EndWhile
		\State res $\gets$ ls \Comment O(1)
		\medskip
		\Statex \underline{Complejidad:} O(longitud(v))
		\Statex \underline{Justificacion:}Para devolver una lista con los elementos que quedan por recorrer, recorro todo el vector viendo que elementos son válidos y si un elemento es válido lo agrego a una lista.
	\end{algorithmic}
\end{algorithm}

\end{Algoritmos}

\newpage
\section{DiccString($\sigma$)}

\subsection{Interfaz}

\begin{Representacion}
\subsection{Justificacion}
	\begin{Estructura}{diccString}[puntero(nodo($\sigma$))]
		\begin{Tupla}[nodo]
			\tupItem{hijos}{arreglo[256] de puntero(nodo($\sigma$)}
			\tupItem{significado}{puntero($\sigma$)}
		\end{Tupla}
	\end{Estructura}
\subsection{Invariante de representación}

\textbf{Informal}


\subsection{Predicado de abtraccion}

\AbsFc[puntero(nodo($\sigma$))]{DiccString(string, $\sigma$)}[p]{d: DiccString(string, $\sigma$) | ($\forall$ s: string)(Def?(s, $\sigma$) $\Leftrightarrow$ ())}

\end{Representacion}
\newpage
\section{colaPrioridadMin($\sigma$)}

\textbf{parámetros formales}\hangindent=2\parindent\\
\parbox{1.7cm}{\textbf{géneros}}  $\sigma$\\
\parbox[t]{1.7cm}{\textbf{función}}\parbox[t]{\textwidth-2\parindent-1.7cm}{%
\InterfazFuncion{$\bullet$=$\bullet$}{\In{s1, s2}{$\sigma$}}{$\sigma$}
{$res \igobs s1 = s2$}
[$O(1)$]
[función de igualdad de $\sigma$'s]
}

\subsection{Interfaz}

\InterfazFuncion{Vacia}{}{colaPrioridadMin($\sigma$)}
{res = vacia}
[O(1)]
[Crea una cola de prioridad vacia]\\

\InterfazFuncion{EsVacia?}{\In {c}{colaPrioridadMin($\sigma$)}}{bool}
{$res \igobs vacia?(c)$}
[O(1)]
[Devuelve verdadero si la lista es vacia.]\\

\InterfazFuncion{Proximo}{\In {c}{colaPrioridadMin($\sigma$)}}{$\sigma$}
[$\neg EsVacia?(x)$]
{$res \igobs proximo(c)$}
[O(1)]
[Devuelve el valor del primer elemento de la cola de prioridad]\\

\InterfazFuncion{Encolar}{\Inout {c}{colaPrioridadMin($\sigma$)}, \In {k}{$\kappa$}, \In {s}{$\beta$}}{iterColaMin}
{$res \igobs encolar(x,c)$}
[O(log(n))]
[Agrega un elemento a la cola de prioridad. Devuelve un iterador apuntando al elemento insertado]\\

\InterfazFuncion{Desencolar}{\Inout {c}{colaPrioridadMin($\sigma$)}}{}
[$\neg EsVacia?(x)$]
{$res \igobs desencolar(c)$}
[O(log(n))]
[Elimina el proximo elemento de la cola de prioridad.]\\

% FALTA BORRAR DEL ITERADOR

\begin{Representacion}

\subsection{Justificacion}
Implementamos la cola de prioridad sobre minHeap para poder tener acceso en O(1) al primer de elemento de la cola y tener las operaciones de encolar y desencolar en O(log(n)). $\newline$ Como queremos poder agregar y quitar una arbitraria cantidad de elementos, no podemos usar un arreglo para representar el heap, dado que para extender el arreglo tendríamos una complejidad mayor a la pedida. $\newline$ Lo representamos sobre un árbol binario completo izquierdista, cumpliendo la complejidad de heap. $\newline$ El principal uso de la cola de prioridad será para determinar qué jugador captura al pokemon, la prioridad será: aquel que tenga la menor cantidad de pokemons tendrá la mayor prioridad. $\newline$ No nos interesa buscar en la estructura de manera eficiente (la complejidad de la búsqueda es O(n)) pero sí nos interesa obtener el primer elemento en O(1). $\newline$
	\begin{Estructura}{colaPrioridadMin}[tupla(raiz: puntero(nodoHeap($\sigma$)), padreAgregar: puntero(nodoHeap($\sigma$)))]
		\begin{Tupla}[nodoHeap]
			\tupItem{padre}{puntero(nodoHeap($\sigma$))}
			\tupItem{izq}{puntero(nodoHeap($\sigma$))}
			\tupItem{der}{puntero(nodoHeap($\sigma$))}
			\tupItem{clave}{$\kappa$}
			\tupItem{significado}{$\beta$}
		\end{Tupla}
	\end{Estructura}
\subsection{Invariante de representación}

\textbf{Informal}\\
(1)Para todo nodo, sus hijos no pueden tenerlo como hijo.\\
(2)Todas las hojas tienen significado.\\
(3)La raíz no tiene padre.\\
(4)Árbol binario perfectamente balanceado e izquierdista.\\
(5)La clave de cada nodo es menor o igual a la de sus hijos (si los tiene).\\
(6)Todo subárbol es un heap.\\

\subsection{Predicado de abtraccion}

%\AbsFc[puntero(nodo)]{dicc(string, $\sigma$)}[p]{ d : dicc(string,$\sigma$) | $(\forall s:string)\Big( \big(Def?(s,d)$ $\Longleftrightarrow (encontrarPalabra(s,p) \neq NULL$ $\yluego$ $encontrarPalabra(s,p)\to significado \neq NULL)\big)$ $\land$ $\big(*(encontrarPalabra(s,p)\to significado) = obtener(s,p)\big)\Big)$}


%~  
% \tadOperacion{encontrarPalabra}{string,puntero(nodo)}{puntero(nodo)}{}
%  \tadAxioma{encontrarPalabra($s$, $p$)}{\IF $vacia(s)$ $\lor$ $p=NULL$  THEN $p$ ELSE encontrarPalabra($fin(s)$, $p\to caracteres$[$ord(prim(s))$]) FI}

\subsection{Algoritmos}

\begin{Algoritmos}

\begin{algorithm}[H]{\textbf{iVacia}() $\to$ res: puntero(nodoHeap($\sigma$))}
	\begin{algorithmic}[1]
		%\State a $\gets$ puntero(nodoHeap($\sigma$)) \Comment O(1), creo un nodo vacío
		\State $res \gets NULL$ \Comment O(1)
		
		\medskip
		\Statex \underline{Complejidad:} O(1)
			\Statex \underline{Justificacion:} Crea un heap vacio, como siempre va a tener que crear una raíz vacía, la complejidad queda constante.
	\end{algorithmic}
\end{algorithm}


\begin{algorithm}[H]{\textbf{iEsVacia?}(\In {r}{puntero(nodoHeap($\sigma$))}) $\to$ res: bool)}
	\begin{algorithmic}[1]
		\State $res \gets r = NULL$\Comment O(1)
		
		\medskip
		\Statex \underline{Complejidad:} O(1)
			\Statex \underline{Justificacion:} Hace una comparación con la raíz para verificar que la cola de prioridad esté vacía.
	\end{algorithmic}
\end{algorithm}

\begin{algorithm}[H]{\textbf{iEncolar}(\Inout {c}{colaPrioridadMin($\sigma$)}, \In {k}{$\kappa$}, \In {s}{$\beta$}) $\to$ res: iterColaMin}
	\begin{algorithmic}[1]

		% Insertamos el elemento
		\State BuscarPadreAgregar(c) \Comment O(log(n))
		\,
		\ puntero(nodoHeap($\sigma$)) nuevo $\leftarrow$ c.padreAgregar$\rightarrow$izq\Comment O(1)
		\,
		\State nuevo$\rightarrow$izq $\leftarrow$ NULL\Comment O(1)
		\State nuevo$\rightarrow$der $\leftarrow$ NULL\Comment O(1)
		\State nuevo.clave $\leftarrow$ k\Comment O(1)
		\State nuevo.significado $\leftarrow$ s\Comment O(1)
		\State nuevo$\rightarrow$padre $\leftarrow$ c.padreAgregar\Comment O(1)
		\,	
		\If{c.padreAgregar$\rightarrow$izq = NULL}\Comment O(1)
			\State c.padreAgregar$\rightarrow$izq $\leftarrow$ nuevo\Comment O(1)
		\,
		\Else
			\State c.padreAgregar$\rightarrow$der $\leftarrow$ nuevo\Comment O(1)
		\EndIf
		\,
		% Sift up
				
		\While{nuevo.clave < nuevo$\rightarrow$padre.clave}\Comment O(log(n))
			\State siftUp(c, nuevo)\Comment O(1)
		\EndWhile			
		\,		
		\State res$\rightarrow$nuevo
		
		
		\medskip
		\Statex \underline{Complejidad:} O(log(n))
			\Statex \underline{Justificacion:} Se agrega un elemento al fondo del heap, luego se utiliza la función sift-up para volver a la propiedad de heap. Ocurren varias operaciones independientes que tienen complejidad O(log(n)), por lo tanto tiene una complejidad O(log(n)).
	\end{algorithmic}
\end{algorithm}


\begin{algorithm}[H]{\textbf{iDesencolar}(\Inout {c}{colaPrioridadMin($\sigma$)}}
	\begin{algorithmic}[1]

		\State puntero(nodoHeap($\sigma$)) ultimo $\leftarrow$ NULL\Comment O(1)
		\,
		\If{c.padreAgregar$\rightarrow$der = NULL}\Comment O(1)
			\State ultimo $\leftarrow$ c.padreAgregar$\rightarrow$izq\Comment O(1)
			\State c.padreAgregar$\rightarrow$izq$\leftarrow$NULL\Comment O(1)
			\,
		\Else
			\State ultimo $\leftarrow$ c.padreAgregar$\rightarrow$der\Comment O(1)
			\State c.padreAgregar$\rightarrow$der$\leftarrow$NULL\Comment O(1)
		\EndIf
		\State ultimo$\rightarrow$izq$\leftarrow$c.raiz$\rightarrow$izq\Comment O(1)
		\State ultimo$\rightarrow$der$\leftarrow$c.raiz$\rightarrow$der\Comment O(1)
		\State c.raiz$\leftarrow$ultimo\Comment O(1)
		\,
		%Ahora sift down
		\While {ultimo$\rightarrow$der $\neq$ NULL $\wedge$ ultimo$\rightarrow$izq $\neq$ NULL $\wedge$ (ultimo$\rightarrow$der.clave $<$ ultimo.clave $\vee$ ultimo$\rightarrow$izq.clave $<$ ultimo.clave))} \Comment O(log(n))
			\State siftDown(ultimo, c)\Comment O(1)
		\EndWhile
		
		\medskip
		\Statex \underline{Complejidad:} O(log(n))
			\Statex \underline{Justificacion:} Se elimina el primer elemento del heap, se reemplaza la raiz por el último elemento, luego se utiliza la función sift-down para volver a tener la propiedad de heap. El ciclo itera como máximo log(n) veces, que es la altura del heap (sólo se puede llevar hasta el fondo).  
	\end{algorithmic}
\end{algorithm}


\begin{algorithm}[H]{\textbf{iProximo}(\In {c}{colaPrioridadMin($\sigma$)}) $\to$ res: $\sigma$}
	\begin{algorithmic}[1]
		
		\State res$\leftarrow$c$\rightarrow$raiz.significado
		
		\medskip
		\Statex \underline{Complejidad:} O(1)
			\Statex \underline{Justificación:} Se obtiene el significado de la raiz del heap.
	\end{algorithmic}
\end{algorithm}


\begin{algorithm}[H]{\textbf{iBorrarConIt}(\Inout {c}{colaPrioridadMin($\sigma$)}, \Inout {p}{iterCola}}
	\begin{algorithmic}[1]
		\State puntero(nodoHeap($\sigma$)) ultimo $\leftarrow$ NULL\Comment O(1)
		\,
		\If{c.padreAgregar$\rightarrow$der = NULL}\Comment O(1)
			\State ultimo$\leftarrow$ c.padreAgregar$\rightarrow$izq\Comment O(1)
			\State c.padreAgregar$\rightarrow$izq$\leftarrow$NULL\Comment O(1)
			\,
		\Else
			\State ultimo $\leftarrow$ c.padreAgregar$\rightarrow$der\Comment O(1)
			\State c.padreAgregar$\rightarrow$der$\leftarrow$NULL\Comment O(1)
		\EndIf
		\State ultimo$\rightarrow$izq$\leftarrow$p$\rightarrow$izq\Comment O(1)
		\State ultimo$\rightarrow$der$\leftarrow$p$\rightarrow$der\Comment O(1)
		\State ultimo$\to$padre$\leftarrow$p$\to$padre	\Comment O(1)
		\,
		%Ahora sift down
		\While {ultimo$\rightarrow$der $\neq$ NULL $\wedge$ ultimo$\rightarrow$izq $\neq$ NULL $\wedge$ (ultimo$\rightarrow$der.clave $<$ ultimo.clave $\vee$ ultimo$\rightarrow$izq.clave $<$ ultimo.clave))} \Comment O(log(n))
			\State siftDown(ultimo, c)\Comment O(1)
		\EndWhile
		
		\medskip
		\Statex \underline{Complejidad:} O(log(n))
			\Statex \underline{Justificacion:} Muy similar a iDesencolar(), pero eliminando el puntero del parámetro. $\newline$ Se elimina el elemento apuntado por el iterador, se reemplaza por el último elemento, luego se utiliza la función sift-down para volver a tener la propiedad de heap. El ciclo itera como máximo log(n) veces, que es la altura del heap (sólo se puede llevar hasta el fondo).  
	\end{algorithmic}
\end{algorithm}



%%%%%%%%%%%%%%%%%%%%%%%%%%%%%%
% Funciones auxiliares
%%%%%%%%%%%%%%%%%%%%%%%%%%%%%%

\begin{algorithm}[H]{\textbf{BuscarPadreInsertar}(\In {c}{colaPrioridadMin($\sigma$)})}
	\begin{algorithmic}[1]
		
		\If {c.padreAgregar$\rightarrow$der $\neq$ NULL}\Comment O(1)
			\State puntero(nodoHeap($\sigma$)) aux $\leftarrow$ c.padreAgregar\Comment O(1)
			\,	
			\While {aux $\neq$ c.raiz $\wedge$ aux$\rightarrow$padre$\rightarrow$der = aux}\Comment O(log(n))\,
				\Comment Buscamos el padre hacia la izquierda hasta no poder avanzar
				\State aux $\leftarrow$ aux$\rightarrow$padre\Comment O(1)		
			\EndWhile
			\,			
			\If {aux $\neq$ c.raiz}\Comment O(1)\,
				\Comment Nos movemos al subárbol hermano
				\State aux $\leftarrow$ aux$\rightarrow$padre\Comment O(1)
				\State aux $\leftarrow$ aux$\rightarrow$der\Comment O(1)
			\EndIf
			\,			
			\While {aux$\rightarrow$izq $\neq$ NULL}\Comment O(1)\,
				\Comment Bajamos a la izquierda hasta no poder avanzar.
				\State aux $\leftarrow$ aux$\rightarrow$izq\Comment O(1)
			\EndWhile
			\,
			\State c.padreAgregar $\leftarrow$ aux\Comment O(1)
		\EndIf
		
		\medskip
		\Statex \underline{Complejidad:} O(log(n))
			\Statex \underline{Justificacion:}Se obtiene nodo al cual hay que agregarle hojas para mantener la propiedad de heap al encolar.\,
			     Subimos hacia la izquierda mientras los subárboles estén completos para encontrar el subárbol "hermano" al cual hay que agregarle el elemento a encolar. Luego bajamos hacia la izquierda para encontrar el padre de la futura hoja. En caso de que el árbol total sea completo se sube hasta la raíz y se baja hacia el elemento más a la izquierda.
			\,   En el peor caso se recorre dos veces la altura del heap, es decir 2log(n) operaciones, dando una complejidad O(log(n)). 
	\end{algorithmic}
\end{algorithm}


\begin{algorithm}[H]{\textbf{siftUp}(\In {c}{colaPrioridadMin($\sigma$)}, \In {n}{puntero(nodoHeap($\sigma$))})}
	\begin{algorithmic}[1]
		
		\State puntero(nodoHeap($\sigma$)) aux $\leftarrow$ NULL\Comment O(1)
		\Comment Se crea un nodo auxiliar para poder hacer el intercambio de posicion entre los nodos
		\,
		\If {n$\rightarrow$der.clave < n.clave}\Comment O(1)
			\State aux$\rightarrow$der $\leftarrow$ n$\rightarrow$der$\rightarrow$der\Comment O(1)
			\State aux$\rightarrow$izq $\leftarrow$ n$\rightarrow$der$\rightarrow$izq\Comment O(1)
			\State aux$\rightarrow$padre $\leftarrow$ n$\rightarrow$der\Comment O(1)
			\,
			\State n$\rightarrow$der$\rightarrow$der $\leftarrow$ n$\rightarrow$der\Comment O(1)
			\State n$\rightarrow$der$\rightarrow$izq $\leftarrow$ aux\Comment O(1)
			\State n$\rightarrow$der$\rightarrow$padre $\leftarrow$ n$\rightarrow$padre\Comment O(1)
			\,
		\Else
			\State aux$\rightarrow$der $\leftarrow$ n$\rightarrow$izq$\rightarrow$der\Comment O(1)
			\State aux$\rightarrow$izq $\leftarrow$ n$\rightarrow$izq$\rightarrow$izq\Comment O(1)
			\State aux$\rightarrow$padre $\leftarrow$ n$\rightarrow$izq\Comment O(1)
			\,
			\State n$\rightarrow$izq$\rightarrow$der $\leftarrow$ n$\rightarrow$der\Comment O(1)
			\State n$\rightarrow$izq$\rightarrow$izq $\leftarrow$ aux\Comment O(1)
			\State n$\rightarrow$izq$\rightarrow$padre $\leftarrow$ n$\rightarrow$padre\Comment O(1)
			\,		
		\EndIf
		\,
		\State n $\leftarrow$ aux\Comment O(1)
		
		\medskip
		\Statex \underline{Complejidad:} O(1)
			\Statex \underline{Justificacion:}Se intercambia un nodo con uno de sus hijos que tenga clave menor. Pese a que no tiene precondiciones, es llamado en un contexto en el cual no se puede indefinir. Es una secuencia de asignaciones, todas O(1), por lo tanto el algoritmo tiene complejidad O(1).
	\end{algorithmic}
\end{algorithm}



\begin{algorithm}[H]{\textbf{siftDown}(\In {c}{colaPrioridadMin($\sigma$)}, \In {n}{puntero(nodoHeap($\sigma$))})}
	\begin{algorithmic}[1]
		
		\State puntero(nodoHeap($\sigma$)) aux $\leftarrow$ NULL\Comment O(1) $\,$
		\Comment Se crea un nodo auxiliar para poder hacer el intercambio de posicion entre los nodos
		
		\If {n$\rightarrow$padre$\rightarrow$izq = n}\Comment O(1)
			\State aux$\rightarrow$der $\leftarrow$ n$\rightarrow$padre$\rightarrow$der\Comment O(1)
			\State aux$\rightarrow$izq $\leftarrow$ n$\rightarrow$padre\Comment O(1)
			
		\Else
			\State aux$\rightarrow$der $\leftarrow$ n$\rightarrow$padre\Comment O(1)
			\State aux$\rightarrow$izq $\leftarrow$ n$\rightarrow$padre$\rightarrow$izq\Comment O(1)
		\EndIf
				
			
			
		\If {n$\rightarrow$padre = c.raiz}\Comment O(1)
			\State aux$\rightarrow$padre $\leftarrow$ NULL\Comment O(1)
		\Else
			\State aux$\rightarrow$padre $\leftarrow$ n$\rightarrow$padre$\rightarrow$padre\Comment O(1)
		\EndIf
		
		\State n$\rightarrow$padre$\rightarrow$der $\leftarrow$ n$\rightarrow$der\Comment O(1)
		\State n$\rightarrow$padre$\rightarrow$izq $\leftarrow$ n$\rightarrow$izq\Comment O(1)
		\State n$\rightarrow$padre$\rightarrow$padre $\leftarrow$ aux\Comment O(1)
		
		\State n $\leftarrow$ aux\Comment O(1)
		
		\medskip
		\Statex \underline{Complejidad:} O(1)
			\Statex \underline{Justificacion:}Se intercambia un nodo con su padre. Pese a que no tiene precondiciones, es llamado en un contexto en el cual no se puede indefinir. Es una secuencia de asignaciones, todas O(1), por lo tanto el algoritmo tiene complejidad O(1). 
	\end{algorithmic}
\end{algorithm}



\end{Algoritmos}

\end{Representacion}
\end{document}
