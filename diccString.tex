\section{DiccString($\sigma$)}

\subsection{Interfaz}

\begin{Representacion}
\subsection{Justificacion}
	\begin{Estructura}{diccString}[puntero(nodo($\sigma$))]
		\begin{Tupla}[nodo]
			\tupItem{hijos}{arreglo[256] de puntero(nodo($\sigma$))}
			\tupItem{padre}{puntero(nodo($\sigma$))}
			\tupItem{significado}{puntero($\sigma$)}
		\end{Tupla}
	\end{Estructura}
\subsection{Invariante de representación}

\textbf{Informal}
(1)El puntero(nodo($\sigma$)) no tiene padre ya que es la raiz
(2)Los hijos de cada nodo (para cada elemento del arreglo) tienen como padre a este

\textbf{Formal}
\Rep[puntero(nodoAB) p][p]{(1)p$\to$padre = NULL $\wedge$ (2)($\forall$ $p_1$:puntero(nodo))(esDecendiente($p_1$, p) $\yluego$ ($\forall$ $p_2$: (puntero(nodo))(esHijo?($p_1$, $p_2$, 255)) $\Rightarrow$ $p_2 \to$padre = $p_1$))}



\end{Representacion}