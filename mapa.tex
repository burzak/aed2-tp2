\section{Mapa}


\subsection{Interfaz}

\parbox {1,7cm}{{\bf Género}} mapa\\
{\bf se explica con:}  \tadNombre{Mapa}, \tadNombre{bool}\\
\medskip

\InterfazFuncion{CrearMapa}{}{Mapa}
{res = crearMapa()}
[O(1)]
[Crea un nuevo mapa]
\\

\InterfazFuncion{Coordenadas}{\In {m}{Mapa}}{Conj(Coordenada)}
{res = coordenadas(m)}
[O(1)]
[Devuelve todas las coordenadas del mapa]
\\

\InterfazFuncion{AgregarCoordenada}{\In {c}{Coordenada}, \Inout {m}{Mapa}}{}
[m $\igobs$ m$_0$]
{m = agregarCoor(c, m$_0$)}
[O(n)]
[Agrega la coordenada a mapa]
\\

\InterfazFuncion{HayCamino}{\In {c_1}{Coordenada}, \In {c_2}{Coordenada}, \In {m}{Mapa}}{Bool}
[$c_1 \in coordenadas(m) \wedge c_2 \in coordenadas(m)$]
{res $\igobs$ hayCamino($c_1, c_2$)}
[O(1)]
[Te dice si dos coordenadas estan conectadas]
\\


\InterfazFuncion{PosExistente}{\In {c}{Coordenada},\In {m}{Mapa}}{Bool}
{res $\igobs$ posExistente(c, m)}
[O(1)]
[Devuelve true si existe esa coordenada]
\\


\InterfazFuncion{CoordenadasConectadasA}{\In {c}{Coordenada}, \In {m}{Mapa}}{Conj(Coordenada)}
[$c \in coordenadas(m)$]
{$(\forall c_1: Coordenada) c_1 \in res \wedge c_1 \in coordenadas(m) \impluego hayCamino(c, c_1, m)$}
[O(n)]
[Devuelve un conjunto de coordenadas a las cuales hayCamino]
\\


\begin{Representacion}
\subsubsection{Justificación}
	\begin{Estructura}{Mapa}[infomapa]
		\begin{Tupla}[infomapa]
			\tupItem{coordenadas}{conj(Coordenada)}\\
			\tupItem{relacionCoordenadas}{arreglo(arreglo(Bool))}
			\tupItem{ancho}{Nat}
			\tupItem{alto}{Nat}
		\end{Tupla}
	\end{Estructura}

\subsubsection{Invariante de representación}

\textbf{Informal}\\

Vale para todo par de natruales

\textbf{Formal}\\

\Rep[casillero e][e]{true}

\subsubsection{Predicado de abtraccion}

\AbsFc[estr]{Coordenada}[e]{($\forall$ s:casillero)(Abs(s) $\igobs$ c:Coordenada) | (s.latitud = latitud(c) $\wedge$ s.longitud = longitud(c))}

\end{Representacion}

\subsection{Algoritmos}
\begin{Algoritmos}

\begin{algorithm}[H]{\textbf{iCrearMapa}() $\to$ res: mapainfo}
	\begin{algorithmic}[1]
		\State $res \gets <Vacio(), arreglo[0], 0, 0>$ \Comment O(1)

		\medskip
		\Statex \underline{Complejidad:} O(1)
			\Statex \underline{Justificacion:} Sólo realiza una asignación y las funciones de Vac\'io() de m\'odulo Conjunto Lineal y Diccionario Lineal son O(1)
	\end{algorithmic}
\end{algorithm}

\begin{algorithm}[H]{\textbf{iHayCamino}(\In {c1}{coordenada}, \In {c2}{coordenada}, \In {m} {infomapa}) $\to$ res: bool}
	\begin{algorithmic}[1]
		\State pos1 $\gets$ m.ancho * Longitud(c1) + m.alto * Latitud(c1) \Comment O(1)
		\State pos2 $\gets$ m.ancho * Longitud(c2) + m.alto * Latitud(c2) \Comment O(1)
		\State res $\gets$ m.relacionCoordenadas[pos1][pos2] \Comment O(1)

		\medskip
		\Statex \underline{Complejidad:} O(1)
			\Statex \underline{Justificacion:} Son solamente 3 asignaciones y un acceso de orden de 1 en un arreglo estatico
	\end{algorithmic}
\end{algorithm}

\begin{algorithm}[H]{\textbf{iPosExistente}(\In {c}{coordenada}, \In {m} {infomapa}) $\to$ res: bool}
	\begin{algorithmic}[1]
		\State pos $\gets$ m.ancho * Longitud(c) + m.alto * Latitud(c) \Comment O(1)
		\State res $\gets$ m.relacionCoordenadas[pos][pos] \Comment O(1)

		\medskip
		\Statex \underline{Complejidad:} O(1)
			\Statex \underline{Justificacion:} Es solamente 2 asignaciones y un acceso de orden 1 a un arreglo. Esto funciona porque cuando calculo las relaciones entre las coordenas siempre definimos que una coordenada esta relacionada consigo misma.
	\end{algorithmic}
\end{algorithm}

\begin{algorithm}[H]{\textbf{iAgregarCoordenada}(\In {c}{coordenada}, \In {m}{infomapa})}
	\begin{algorithmic}[1]
		\State \IF Longitud(c) > infomapa.ancho THEN m.ancho $\gets$ Longitud(c) ELSE FI \Comment O(1)
		\State \IF Latitud(c) > infomapa.alto THEN m.alto $\gets$ Latitud(c) ELSE FI \Comment O(1)
		\State Agregar(m.coordenadas, c) \Comment O($\#$m.coordenadas)
		\State m.relacionCoordenadas $\gets$ arreglo[m.ancho*m.alto] de arreglo[m.ancho*m.alto] de Bool \Comment O($(m.ancho*m.alto)^2$)


		\State iter $\gets$ CrearIt(m.coordenadas) \Comment O(1)
		
		\While{HaySiguiente(iter)} \Comment O($\#$m.coordenadas$^3$)
			\State coor $\gets$ Siguiente(iter) \Comment O(1)
			\State Avanzar(iter) \Comment O(1)
			\State conectadas $\gets$ iCoordenadasConectadas(coor, m) \Comment O($\#$m.coordenadas$^2$)
			\State iterConectadas $\gets$ CrearIt(conectadas) \Comment O(1)
			\While {HaySiguiente(iterConectadas)} \Comment O($\#$m.coordenadas)
				\State $coor_2$ $\gets$ Siguiente(iterConectadas) \Comment(1)
				\State Avanzar(iterConectadas) \Comment O(1)
				\State pos1 $\gets$ m.ancho * Longitud(coor) + m.alto * Altitud(coor) \Comment O(1)
				\State pos2 $\gets$ m.ancho * Longitud($coor_2$) + m.alto * Altitud($coor_2$) \Comment O(1)
				\State m.relacionCoordenadas[pos1][pos2] $\gets$ True \Comment O(1)
				\State m.relacionCoordenadas[pos2][pos1] $\gets$ True \Comment O(1)
			\EndWhile
		\EndWhile
		
		\medskip
		\Statex \underline{Complejidad:} O(max($n^3$, $T^2$))
		\Statex \underline{Justificacion:} Donde T es el tama\~no de la grilla de todo el mapa (ancho * alto) y n es el cardinal de coordenadas en el Mapa. Ya que la creaci\'on de los arreglos no es gratis, tiene un costo que es el tama\~no del ancho*alto del Mapa. Tambi\'en ejecutamos un While de n iteraciones donde ejecutamos operaciones que cuestan como m\'aximo $n^2$ por lo cual el While tiene un costo del orden de $n^3$. Dado que la creaci\'on podr\'ia tomar m\'as tiempo que ejecutar el While debemos tomar el m\'aximo valor de ambos como la complejidad del algoritmo.
	\end{algorithmic}
\end{algorithm}

\begin{algorithm}[H]{\textbf{iCoordenadasConectadasA}(\In {c}{coordenada}, \In {m}{infomapa}) $\to$ res: Conj(coordenada)}
	\begin{algorithmic}[1]
		\State visitadas $\gets$ Vac\'io() \Comment O(1)
		\State aVisitar $\gets$ Encolar(Vac\'ia(), c) \Comment  O(1)
		\State res $\gets$ Agregar(Vac\'io(), c) \Comment O(1)
		\While{$\neg$ EsVac\'ia(aVisitar)} \Comment O($\#$m.coordenadas$^2$)
			\State coor $\gets$ Proximo(aVisitar) \Comment O(1)
			\State Desencolar(aVisitar)
			\State Agregar(visitadas, coor) \Comment O($\#$m.coordenadas)			
			\If {Latitud(coor) > 0} \Comment O(1)
					\State coorAbajo $\gets$ CoordenadaAbajo(coor) \Comment O(1)
					\If {$\neg$ Pertenece?(visitadas, coorAbajo) $\wedge$ Pertenece?(m.coordenadas, coorAbajo)} \Comment O($\#$m.coordenadas)
						\State Agregar(res, coorAbajo) \Comment O($\#$m.coordenadas)
						\State Encolar(aVisitar, coorAbajo) \Comment O(copy(coordenada))
					\Else
					\EndIf
			\Else
			\EndIf
			\If {longitud(coor) > 0} \Comment O(1)
					\State coorIzq $\gets$ CoordenadaIzquierda(coor) \Comment O(1)
					\If {$\neg$ Pertenece?(visitadas, coorIzq) $\wedge$ Pertenece?(m.coordenadas, coorIzq)} \Comment O($\#$m.coordenadas)
						\State Agregar(res, coorIzq) \Comment O($\#$m.coordenadas)
						\State Encolar(aVisitar, coorIzq) \Comment O(copy(coordenada))
					\Else
					\EndIf
			\Else
			\EndIf
			\State coorDer $\gets$ CoordenadaDerecha(coor) \Comment O(1)
			\If {$\neg$ Pertenece?(visitadas, coorDer) $\wedge$ Pertenece?(m.coordenadas, coorDer)} \Comment O($\#$m.coordenadas)
				\State Agregar(res, coorDer) \Comment O($\#$m.coordenadas)
				\State Encolar(aVisitar, coorDer) \Comment O(copy(coordenada))
			\Else
			\EndIf
			\State coorArriba $\gets$ CoordenadaDerecha(coor) \Comment O(1)
			\If {$\neg$ Pertenece?(visitadas, coorArriba) $\wedge$ Pertenece?(m.coordenadas, coorArriba)} \Comment O($\#$m.coordenadas)
				\State Agregar(res, coorArriba) \Comment O($\#$m.coordenadas)
				\State Encolar(aVisitar, coorArriba) \Comment O(copy(coordenada))
			\Else
			\EndIf
		\EndWhile

		\medskip
		\Statex \underline{Complejidad:} O($n^2$)
			\Statex \underline{Justificacion:} Dado un mapa y una coordenada te devuelve todas las coordenadas conectadas a esa coordenada inicial. Tomando como \textbf{n} el cardinal de infomapa.coordenadas nos da O($n^2$).
	\end{algorithmic}
\end{algorithm}


\end{Algoritmos}
